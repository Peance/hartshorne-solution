%\documentclass[11pt]{article}          % required in all documents
\documentclass[12pt]{amsart}           % required in all documents
\usepackage{epsfig}                     % a package to include PS figures
\usepackage{amscd}
\newcommand{\tuborg}{\left\{\begin{array}{ll}}
\newcommand{\sluttuborg}{\end{array}\right.}
 
  
\setlength{\textwidth}{6in}             % Space saving measures
\setlength{\textheight}{9in}
\setlength{\topmargin}{-.5in}
\renewcommand{\baselinestretch}{1}
\setlength{\oddsidemargin}{.25in}
\setlength{\evensidemargin}{.25in}
   
\DeclareSymbolFont{AMSb}{U}{msb}{m}{n}
\DeclareMathSymbol{\N}{\mathbin}{AMSb}{"4E}
\DeclareMathSymbol{\Z}{\mathbin}{AMSb}{"5A}
\DeclareMathSymbol{\R}{\mathbin}{AMSb}{"52}
\DeclareMathSymbol{\Q}{\mathbin}{AMSb}{"51}
\DeclareMathSymbol{\I}{\mathbin}{AMSb}{"49}
\DeclareMathSymbol{\C}{\mathbin}{AMSb}{"43}                                    

\begin{document}

\begin{center}
\bf
\large Algebraic Geometry by Robin Hartshorne\\
\end{center}
\begin{center}
Exercises solutions by Jinhyun Park\\
\end{center}
\vskip1cm
Warning!!!!!!\\
1)This material is not for sale.\\
2)This is just for personal use only.\\
3)No commercial purpose please.\\
4)Never Never Ever read this solution unless you tried problems quite long time and gave up. Doing so may impair your ability to think and solve problems.\\

\vskip1.5cm

\subsection*{ Chapter 4.Curves, Section 1.Riemann-Roch Theorem}

\subsubsection*{ 1.} 
Choose $Q \in C$. Choose $n$ big enough so that deg $n(2P-Q) > 2g-2,\ g,\ 1$. $\Rightarrow h^0 (n(2P-Q)) = 1-g+n(2P-Q) >1 \Rightarrow \exists$ effective divisor $D \in |n(2P-Q)| \Rightarrow \exists f \in K(C)$ such that $D+nQ-2nP = (f)$. Since deg $D=n$, so $D$ cannot cancel $-2nP$ i.e. $f$ has a pole only at $P$.//

\subsubsection*{ 2.}

 Let $F=\{P_1 , \cdots , P_r\}$. Multiplying functions of Ex.IV.1.1 might give cancellation of poles and zeros, so we need slightly different approach.\\
Let $Q \in C-F$. Consider $D'=n(P_1 + \cdots P_r - (r-1)Q)$. Choose $n > 2g-2,\ g$. Then, $\exists D \in |D'|$ i.e. $\exists f\in K(c)$ such that $D+(r-1)Q - nP_1 - \cdots -nP_r = (f)$. Note that deg $D = n$. Each $P_i$ occurs with order $-n$ so, if $P_i \in {\rm Supp} D$, then either (i) ${\rm ord}_{P_i} D <n$ or (ii) $D= nP_i$ for some $i$, in this case, WMA $i=1$ WLOG.
\begin{quote}
For (i) there is no problem.\\
For (ii) $f$ has ples only at $P_2, \cdots, P_r$ not at $P_1$. By Ex.IV.1.1, $\exists g \in K(C)$ which has a pole only at $P_1$. Let ${\rm ord}_{P_i} g = n_i \ 2 \leq i \leq r$. Then, if we choose $m > 1, n_2, \cdots, n_r$, then, $f^m g$ has poles at and only at $F$.//
\end{quote}

\subsubsection*{ 3.}

Proof 1) By I-(6.10), there is a projective nonsingular curve $\bar{X}$ over $k$ such that $X$ is an open subset of $\bar{X}$, i.e. $\bar{X} -X$ is a finite set, say, $\{P_1, \cdots, P_r \} \not = \phi$ because $X$ is not proper.\\
Then, by Ex.IV.1.2, $\exists f \in k(\bar{X}) = k(X)$ such that $f$ has poles only at $P_1 , \cdots, P_r$.\\
We can consider $f \in k(\bar{X})$ as a morphism $f : \bar{X} \to \mathbb{P}^1$.Then, $f^{-1} (\mathbb{A}^n) = X$, so, $g=f|_X : X \to \mathbb{A}^1$ is a morphism.\\
$f(\bar(X))$ is proper over $k$(because $\bar{X}$ is proper) and irreducible and, $f(\bar{X}) \not = $a point. Hence, $f(\bar{X}) = \mathbb{P}^1$. And by (II-6.8), $f$ is a finite morphism, in particular, affine morphism. Hence, $f^{-1} (\mathbb{A}^1) =X$ is affine.//\\
\\
Proof 2) As above, let $F=\{P_1 , \cdots , P_r\}= \bar{X} -X$. Choose $m$ such that $mr>2g$ Then, $D = m(P_1 + \cdots P_r$ has a degree $>2g$, so by (3), it is very ample. Then, it gives an embedding of $\bar{X}$ into a projective space $\mathbb{P}^N$ for some $N$, and $D= \bar{X}.H$ for a hyper plane $H$ of  $\mathbb{P}^N$. Then, $\bar{X}-F =$ a closed subscheme of $\mathbb{A}^N = \mathbb{P}^N - H$ which is affine, so, $X=\bar{X}-F$ is also affine.//
\end{document}

