\documentclass[12pt]{article}
\author{William A. Stein}
\title{Algebraic Geometry Homework\\
II.3 \#'s 6,7,8,12,14\\ II.4 \#'s 2, 4}

\font\grmn=eufm10 scaled \magstep 1

\newcommand{\gp}{p}
\newcommand{\ga}{a}
\newcommand{\so}{\mathcal{O}}
\newcommand{\id}{\mbox{\rm id}}

\newtheorem{prob}{Problem}
\newtheorem{prop}{Proposition}
\newtheorem{theorem}{Theorem}

\newcommand{\ov}{\overline{\varphi}}
\newcommand{\ol}{\overline}
\renewcommand{\phi}{\varphi}
\newcommand{\fh}{f^{\#}}
\newcommand{\proj}{\mbox{\rm Proj \hspace{.01in}}}
\newcommand{\spec}{\mbox{\rm Spec \hspace{.01in}}}
\newcommand{\proof}{\mbox{\sc Proof.\hspace{.1in}}}

\newcommand{\diagram}{$$\vspace{.75in}$$}    %% space for diagrams

\date{April 4, 1996}
\begin{document}
\maketitle

\begin{prob}[3.6]
Let $X$ be an integral scheme. Show that the local ring $\so_{\xi}$ of
the generic point $\xi$ of $X$ is a field. Call it $K(X)$. Show also 
that if $U=\spec A$ is any open affine subset of $X$, then $K(X)$ is
isomorphic to the quotient field of $A$.  
\end{prob}
\proof
Let $U=\spec A$ be any nonempty open affine subset of $X$. Then since
the closure of a generic point of $X$ is all of $X$, every open set
must contain a generic point. Thus if $\xi$ is a generic point, then 
$\xi\in U$. But $A$ is an integral domain so $(0)$ is the unique 
generic point of $U$, whence $\xi=(0)$. This shows the generic point is
unique if it exists. Since $X$ is integral it is irreducible so every
open set intersects $U$. Thus every open set contains $(0)\in \spec A=U$, 
so $X$ actually contains a generic point $\xi=(0)$. Furthermore,
$\so_{\xi}\cong A_{(0)}$ is the quotient field of $A$. 

\begin{prob}[3.7]
Let $f:X\rightarrow Y$ be a dominant, generically finite morphism of 
finite type of integral schemes. Show that there is an open dense
subset $U\subseteq Y$ such that the induced morphism 
$f^{-1}(U)\rightarrow U$ is finite. 
\end{prob}
\proof
Let $U=\spec B$ be an open affine subset of $Y$ which contains the
generic point of $Y$. Let $V=\spec A$ be an open affine subset of
$f^{-1}(U)$. Since $f$ is of finite type $A$ is a finitely generated
$B$-algebra. The generic point of $X$ is in $V$ since every
open set contains the generic point. Let $\phi:B\rightarrow A$ be
the homomorphism corresponding to the induced morphism of affine schemes
$f:V\rightarrow U$.
Since $f$ is dominant, we know that $\phi$ is injective. 
The induced map on stalks then gives an inclusion
of function fields $K(Y)=B_{(0)}\hookrightarrow A_{(0)}=K(X)$. 
Since $A$ is a finitely generated $B$-algebra, $K(X)$ is a finitely
generated field extension of $K(Y)$. If this field extension is not of 
finite degree then $K(X)$ must contain an element which is
transcendental over $K(Y)$. Thus $A$ must contain an
element $t$ which is transcendental over $B$. But then infinitely
many primes of $A$ lie over $(0)$. Indeed, since $K(X)$ is
finitely generated over $K(Y)$, $K(X)$ is
a finite algebraic extension of $k(t)$ for some field $k$.
Then (since the algebraic closure of a field is
infinite), there are infinitely many irreducible polynomials
in $k[t]$. Since $K(X)$ is finite algebraic over $k(t)$ 
infinitely many of these must remain irreducible in $A$.  
Multiplying through denominators this gives infinitely
many irreducible elements of $A$ which generate prime ideals
which lie over $(0)$. This would contradict the fact that
$f$ is generically finite. Thus $K(X)$ is finite over $K(Y)$.  

Let $x_1,\ldots,x_n$ generate $A$ as a $B$-algebra. Then,
since $K(X)$ is finite over $K(Y)$ each $x_i$ satisfies
some polynomial $f_i$ with coefficients in $B$. Let $b$ be the
product of all of the leading coefficients of the polynomials
$f_i$. If $b$ is a unit in $B$ then so are all of the leading
coefficients of the $f_i$ so we can divide by them and hence assume
the $f_i$ are monic polynomials. If not, replace $B$ by the localization
$B_b$ and repeat the whole argument with $U=\spec B_b$.  
In either case we may assume the $f_i$ are monic from which
we conclude that $A$ is a finitely generated integral extension
of $B$, thus $A$ is a finite module over $B$. 

Now let $U=\spec B$ be an open affine subset of $Y$ which contains the
generic point. Since $f$ is of finite type we may write
$f^{-1}(U)=\cup_{i=1,\ldots,n}V_i$ where each $V_i=\spec A_i$ is
finitely generated $B$-algebra. By the work above we may shrink
$U$ so that we can assume each $A_i$ is actually a finitely
generated $B$-module. To complete the proof we need to show that
there is a distinguised open subset of $U$ (which necessarily
contains the generic point) whose inverse image under $f$ is an 
open affine which is the spectrum of a finitely generated $B$-module. 
Let $\phi_i:B\rightarrow A_i$ be the homomorphism which
induces $f|_{V_i}$. Since $f$ is dominant, each $\phi_i$
is an injection. Thus we may, for notational convenience,
identify $B$ with it's images in the various $A_i$. The
morphism $f$ is then induced by the inclusion map
$B\hookrightarrow A_i$.  

Since $\cap_{i=1,\ldots,n}V_i$ is open we can,
for each $i$, $1\leq i \leq n-1$, find $\alpha_i\in A_i$
such that $\spec (A_i)_{\alpha_i}\subseteq \cap_{i=1,\ldots,n}V_i$.
Since $A_i$ is a finite module over $B$ there is an
integral equation 
$$\alpha_i^n+b_{n-1}\alpha_i^{n-1}+\cdots+b_0=0$$
where each $b_j\in B$ and $b_0\not=0$. 
Let $b=\prod_{i=1,\ldots,n-1}b_i$. Then any
prime of $A_i$ which contains $\alpha_i$ must
also contain $b_i$ and hence $b$. Therefore
$\spec (A_i)_{b}\subseteq \spec (A_i)_{\alpha}$. 
We then have that
$g^{-1}(\spec B_b) 
= \cup_{i=1,\ldots,n-1}\spec (A_i)_b\cup\spec(A_n)_b
= \spec(A_n)_b$. 
The latter equality follows since, for $1\leq i\leq n-1$, 
$\spec(A_i)_b\subseteq \cap_{i=1,\ldots,n} V_i \cap f^{-1}(U_b)
\subseteq f^{-1}(U_b)\cap V_n = \spec (A_n)_b$.
We thus see that $U_b$ is a dense open
subset of $Y$ such that the morphism $f:f^{-1}(U_b)\rightarrow U_b$
is finite (since $f^{-1}(U_b)$ is the affine scheme $\spec(A_n)_b$
which a finite $B_b$-module). This completes the proof. 

\begin{prob}[3.8] 
Normalization. Let $X$ be an integral scheme. For each open affine
subset $U=\spec A$, let $\tilde A$ be the integral closure of $A$
in its quotient field, and let $\tilde U=\spec \tilde A$.
Show that one can glue the schemes $\tilde U$ to obtain a normal integral
scheme $\tilde X$, called the normalization of $X$. Show also that there
is a morphism $\tilde X\rightarrow X$, having the following universal
property: for every normal integral scheme $Z$, and for every
dominant morphism $f:Z\rightarrow X$, $f$ factors uniquely through
$\tilde X$. If $X$ is of finite type over a field $k$, then the morphism
$\tilde X\rightarrow X$ is a finite morphism. 
\end{prob}
\proof 

We first verify the universal property for affine schemes where
it is clear what the normalization is. 
\begin{prop} Suppose $X=\spec A$, $\tilde X=\spec \tilde A$ 
its normalization and $Z=\spec B$ is a normal integral scheme.   
Then every dominant morphism $f:Z\rightarrow X$ factors uniquely
through $\tilde X$. 
\end{prop}
\proof
Let $\phi:A\rightarrow B$ be the homomorphism corresponding to $f$.  
Then, since $f$ is dominant, $\phi$ is injective. Indeed, 
$f(Z)\subseteq V(\ker(\phi))$ so if $\ker(\phi)\not = 0$
then $f(Z)$ doesn't meet the nonempty open set 
$X-V(\ker(\phi))$ (nonempty since $A$ is a domain
so $(0)$ is prime so $(0)\in X-V(\ker(\phi))$. 
So there is a unique extension of $\phi$ to a 
homomorphism from $\tilde A \rightarrow B$. 
Indeed, define $\ol{\phi}(a/b)=\phi(a)/\phi(b)$. Then
since $\phi$ is injective this is well-defined since $b\not=0$
implies $\phi(b)\not=0$. Furthermore, if $\psi$ is another
possible extension of $\phi$ to $\tilde A$, then 
$\psi(b)\psi(a/b)=\psi(a)=\phi(a)=\phi(b)\ol{\phi}(a/b)=\psi(b)\ol{\phi}(a/b)$
so cancelling shows that $\psi(a/b)=\ol{\phi}(a/b)$.  
Thus there is a unique morphism $f':Z\rightarrow
\tilde X$ whose composition with the natural map $\tilde X\rightarrow X$
equals $f$. (The natural map is induced by the inclusion 
$A\hookrightarrow \tilde A$.)   

Now we prove the universal property holds when $Z$ is an arbitrary
normal integral scheme but $X$ is still affine.  
\begin{prop} Let $X=\spec A$, $\tilde X=\spec \tilde A$ its 
normalization and $Z$ be any normal integral scheme. Then
every dominant morphism $f:Z\rightarrow X$ factors uniquely 
through $\tilde X$. 
\end{prop}
\proof
Let $U_i$ be a cover of $Z$ by open affines. If $U=U_i$ is any
$U_i$ then $U$ is a normal
integral affine scheme and $f|_U$ is a dominant morphism. Indeed,
$U$ is dense in $Z$ since $Z$ is irreducible (Proposition 3.1). 
Thus $f^{-1}(\ol{f(U)})\supseteq \ol{U}=Z$ so 
$f^{-1}(\ol{f(U)})=Z$ so $\ol{f(U)}\supseteq f(Z)$ so
$\ol{f(U)}=\ol{f(Z)}=X$.
We can thus apply the above proposition to find a unique morphism
$g_i:U_i\rightarrow \tilde{X}$ such that $\psi\circ g_i=f|_U$ where
$\psi:\tilde{X}\rightarrow X$. By uniqueness on a cover of $U_i\cap U_j$
by open affines, $g_i|_{U_i\cap U_j}=g_j|_{U_i\cap U_j}$. We
can thus glue the morphism $g_i$ to obtain a morphism 
$g:Z\rightarrow \tilde{X}$ such that $\psi\circ g=f$. The  
morphism $g$ is evidently unique.  

Now we can define the identification maps $\phi_{ij}$. Let
$\{U_i=\spec A_i\}$ be the open affine subsets of $X$. Let
$\{\tilde U_i=\spec \tilde A_i\}$ be the associated normalizations.
Let $\psi_i:\tilde U_i \rightarrow U_i$ be the morphism induced
by the inclusion $A_i \hookrightarrow \tilde A_i$. Let
$W_{ij}=\psi_i^{-1}(U_i\cap U_j)$. Then $W_{ij}$ is an open
subset of a normal scheme hence normal. $\psi_i:W_{ij}\rightarrow
U_i\cap U_j \subseteq U_j$ so there is a unique morphism
which we call $\phi_{ij}:W_{ij}\rightarrow \tilde U_j$ such
that $\psi_j|_{W_{ji}} \circ \phi_{ij} = \psi_i|_{W_{ij}}$.
By uniqueness we see that $\phi_{ij}\circ\phi_{ji}=id$ so
$\phi_{ij}=\phi_{ji}^{-1}$.  Furthermore, for each $i,j,k$, 
$\phi_{ij}(W_{ij}\cap W_{ik})
=\psi_j^{-1}(\psi_i(W_{ij}\cap W_{ik}))
=\psi_j^{-1}(\psi_i(W_{ij}))\cap \psi_j^{-1}(\psi_i(W_{ik}))
=W_{ji}\cap W_{jk}$. By uniqueness, 
$\phi_{jk}\circ\phi_{ij}=\phi_{ik}$ on $W_{ij}\cap W_{ik}$. 
So by the glueing lemma (Exercise 2.12) we may glue to obtain
a scheme $\tilde X$. We can also glue the morphisms $\psi_i$ 
to obtain a morphism $\psi:\tilde X\rightarrow X$. 

Next, we must verify that the universal property holds in general.
Let $Z$ be an arbitrary normal integral scheme, and let $X$
and $\tilde{X}$ be as above and suppose $f:Z\rightarrow X$ is a morphism. 
Cover $X$ be open affines $U_i$. Then for each morphism $f|_{f^{-1}(U_i)}$
we can apply the above proposition to find a morphism $g_i$ such
that $\psi\circ g_i=f|_{f^{-1}(U_i)}$. By uniqueness we can
glue these morphism to obtain the required morphism 
$g:Z\rightarrow \tilde{X}$. 

Now we check that $\tilde{X}$ is a normal integral scheme. 
Note first that each $\tilde{U_i}$ is the spectrum of
an integrally closed domain and is hence a normal integral
scheme (since the localization of an integrally closed
domain is integrally closed). 
Let $x\in\tilde{X}$. Then $x$ is contained in some $\tilde{U_i}$.
But the local ring of $x$ in $\tilde{X}$ is the same as the
local ring of $x$ in $\tilde{U_i}$ which is integrally closed.
This shows that $\tilde{X}$ is normal. 

Since $X$ is irreducible,
every $U_i$ intersects every $U_j$. Thus every $\tilde{U_i}$
intersects every $\tilde{U_j}$ after glueing.  
Since each $\tilde{U_j}$ is irreducible and they all overlap
this implies $\tilde{X}$ is irreducible. Indeed, if $\tilde{X}=A\cup B$
with $A$ and $B$ closed, then every $\tilde{U_i}$ is either completely
contained in $A$ or in $B$. If they are not all contained in one of
$A$ or $B$ then we can find an open subset $U$ contained in $A$ 
and an open subset $V$ contained in $B$ but not contained in $A$. Then 
$V=(U\cap V)\cup(U^c\cap V)
  =(A\cap V)\cup(U^c\cap V)$
which expresses $V$ as a union of two proper closed subsets of $V$.
$A\cap V$ is a proper subset of $V$ since $V$ is not contained in
$A$ and $U^c\cap V$ is a proper subset of $V$ since 
$U\cap V\not=\emptyset$. This contradicts the fact that $V$ is
irreducible. Thus $\tilde{X}=A$ or $\tilde{X}=B$ whence $\tilde{X}$
is irreducible. 

Now we check that the structure sheaf has no nilpotents.
Let $U$ be an open subset of $\tilde{X}$ and suppose
$f\in\so_{\tilde{X}}(U)$ is nilpotent. Then since $f$ is
nonzero, there is some point $x\in \tilde{X}$ so that the
stalk $f_x$ of $f$ at $x$ in the local ring $\so_{\tilde{X}}$ is
nonzero and nilpotent (use sheaf axiom (iii) and the definition of
$\so_{\tilde{X}}$.) Let $\tilde{U_i}$ be some $\tilde{U_i}$ 
which contains $x$. Then the local ring at $x$ is a localization
of the integral domain $\tilde{A_i}$ so it can't contain any
nilpotents. Thus the scheme $\tilde{X}$ is reduced.
  
Now we check that if $X$ is of finite type over a field $k$, then 
the morphism $f:\tilde X\rightarrow X$ is a finite morphism. 
The $U_i$ form an open cover of $X$ and $f^{-1}(U_i)=\spec \tilde A_i$
is affine for each $i$, so we just need to check that
$\tilde A_i$ is a finite module over $A_i$. This follows from
Theorem 3.9A of chapter I.  


\begin{prob}[3.12]
Closed Subschemes of $\proj S$.

(a) Let $\phi:S\rightarrow T$ be a surjective homomorphism of graded
rings, preserving degrees. Show that the open set $U$ of (Ex. 2.14)
is equal to $\proj T$, and the morphism $f:\proj T\rightarrow\proj S$
is a closed immersion.  

(b) If $I\subseteq S$ is a homogeneous ideal, take $T=S/I$ and let $Y$
be the closed subscheme of $X=\proj S$ defined as the image of the
closed immersion $\proj S/I\rightarrow X$. Show that different homogeneous
ideals can give rise to the same closed subscheme. For example,
let $d_0$ be an integer, and let $I'=\bigoplus_{d\geq d_0} I_d$. Show
that $I$ and $I'$ determine the same closed subscheme. 
\end{prob}
\proof
(a) Since $\phi$ is graded and surjective, $\phi(S_{+})=T_{+}$ from which
it is immediate that $U=\proj T$. By the first isomorphism theorem
$T\cong S/\ker(\phi)$ so $f(\proj T)=f(\proj S/\ker(\phi))
=V(\ker(\phi))$ which is a closed subset of $\proj S$. (This is
just the fact that there is a one to one correspondence between
homogeneous ideals of $S/\ker(\phi)$ and homogeneous ideals of $S$ 
which contain $\ker(\phi)$.) The map on the stalk corresponding
to a point $x\in\proj T$ is the map 
$S_{(\phi^{-1}(x))}\rightarrow T_{(x)}$
induced by $\phi$. This map is surjective since $\phi$ is surjective.
Thus the induced map on sheaves is surjective.   

(b) Let $\phi:S/I'\rightarrow S/I$ be the natural projection homomorphism.  
(This makes sense because $S/I$ is a quotient of $S/I'$. Indeed,
$S/I=(S/I')/\bigoplus_{0\leq d<d_0}I_d$.) Then $\phi$ is a graded homomorphism
of graded rings such that $\phi_d$ is the identity for $d\geq d_0$.
So by (Exercise 2.14c) $\phi$ induces an isomorphism
$f:\proj S/I \rightarrow \proj S/I'$. Since this is a morphism
over $\proj S$ (the corresponding triangle of homomorphisms commutes)
it follows that $I$ and $I'$ give rise to the same closed subscheme. 

\begin{prob}[3.14]
If $X$ is a scheme of finite type over a field, show that
the closed points of $X$ are dense. Give an example to show that
this is not true for arbitrary schemes.
\end{prob}
\proof
Since $X$ is of finite type over $k$ we can cover $X$ with
affine open sets $U_i=\spec A_i$ where each $A_i$ is a finitely
generated $k$-algebra. Let $U$ be an open subset of $X$. We must
show that $U$ contains a closed point. Since the $U_i$ cover
$X$, $U$ must intersect some $U_i$. Then $U\cap U_i$ contains
a distinguised open subset of $U_i$. So, to show that every open
set contains a closed point, it suffices to show that every 
nonempty distinguised open subset of each $U_i$ contains 
a closed points of $X$. Since a distinguished open subset $(U_i)_x$ of
a $U_i$ is also the spectrum of a finitely generated $k$-algebra
$\spec (A_i)_x$ we can just add it to our collection $\{U_i\}$. 
The problem thus reduces to showing that each $U_i$ contains
a closed point.

\begin{prop} 
With the notation as above, if $x\in U_i$ is closed in $U_i$ (here
$U_i$ has the subspace topology) then $x$ is closed in $X$. 
\end{prop}
\proof
Suppose $x\in U_j$. There is a natural injection 
$U_i\cap U_j\hookrightarrow U_j$. Let $\spec (B_i)_f$ be
a distinguished open subset of $U_i$ contained in $U_i\cap U_j$
which contains $x$. Then we have a morphism 
$\spec(B_i)_f\hookrightarrow U_j=\spec B_j$. We
thus get a ring homomorphism $\phi:B_j\rightarrow (B_i)_f$
of Jacobson rings. Since it is induced by a restriction of 
the identity map $X\rightarrow X$ which is a morphism over
$k$, $\phi$ is a $k$-algebra homomorphism. 
Since $(B_i)_f$ is a finitely generated $k$-algebra, $(B_i)_f$ is
also a finitely generated $B_j$-algebra. Since $x$ is closed
in $\spec(B_i)_f$, $x$ is a maximal ideal of $(B_i)_f$. 
Thus by page 132 of Eisenbud's {\em Commutative Algebra}
$\phi^{-1}(x)$ is a maximal ideal of $B_j$. Thus $x$ is also
a closed point of $U_j$ in the subspace topology on $U_j$. 
Thus $X-x=\cup_{i}(U_i-x)$ is a union of open subsets of $X$, 
hence open in $X$, so $x$ is closed. 

To finish we just need to know that $U_i$ has a closed point. 
  
\begin{prop}
Let $X=\spec A$ be an affine scheme with $A$ a finitely generated
$k$-algebra. Then any nonempty distinguished open subset of $X$ 
contains a closed point.
\end{prop}
\proof
The key observation is that $A$ is a Jacobson algebra since it finitely
generated over a field, so by page 131 of Eisenbud's {\em Commutative
Algebra} the Jacobson radical of $A$ equals
the nilradical of $A$. Let $D(f)$ be a nonempty
distinguised open subset of $X$. Then some prime omits $f$
so $f$ is not in the nilradical of $A$. Thus $f$ is not
in the Jacobson radical of $A$ so there is some maximal
ideal $m$ so that $f\notin m$. Then $m \in D(f)$ and 
$m$ is a closed point of $X$ since $m$ is maximal.

Strangely enough I never used the hypothesis that $X$ is of
finite type over $k$ but just the weeker hypothesis that
$X$ is {\em locally} of finite type over $k$. Did I miss something? 

Finally, we present a counterexample in the more general situation. Let
$X=\spec Z_{(2)}$. Then $X$ contains precisely one closed
point, the ideal $(2)$. So the set of closed points in $X$ is
not dense in $X$. In fact, if $X$ is the spectrum of any
DVR we also get a counterexample.   

\begin{prob}[4.2]
Let $S$ be a scheme, let $X$ be a reduced scheme over $S$, and let
$Y$ be a seperated scheme over $S$. Let $f_1$ and $f_2$ be two $S$-morphisms
of $X$ to $Y$ which agree on an open dense subset $U$ of $X$. Show that
$f_1=f_2$. Give examples to show that this result fails if either (a)
$X$ is nonreduced, or (b) $Y$ is nonseparated. 
\end{prob}
\proof
Let $g=(f_1,f_2)_S:X\rightarrow Y\times_S Y$ be the product of
$f_1$ and $f_2$ over $S$. By hypothesis the diagonal $T=\Delta_Y(Y)$ is
a closed subscheme of $Y\times_S Y$. Thus $Z=g^{-1}(T)$ is a closed
subscheme of $X$. If $h:U\rightarrow Y$ is the common restriction
of $f_1$ and $f_2$ to $U$, then, since $g|_U$ makes
the correct diagram commute, the restriction of $g$ to $U$ 
is $g'=(h,h)_S$ and $g'=\Delta_Y \circ h$ since 
$\Delta_Y\circ h$ makes the correct diagram commute.
\diagram
Thus $\Delta_Y^{-1}(T)=Y$ implies 
$$g^{-1}(T)\supseteq (g')^{-1}(T)=h^{-1}(\Delta_Y^{-1}(T))=h^{-1}(Y)=U.$$ 
Thus $g^{-1}(T)$ is a closed set which contains the dense
set $U$. Thus $g^{-1}(T)=X$ so $g(X)\subseteq T$. 
So, by the proposition below, since $X$ is reduced, $g$
factors as $g=\Delta_Y\circ f$ where $f:X\rightarrow Y$. 
From the definition of $\Delta_Y$, we see that
$\pi_1\circ \Delta_Y=\id_Y=\pi_2\circ \Delta_Y$. 
Thus $f_1=\pi_1\circ g=\pi_1\circ\Delta_Y\circ f=f$
and $f_2=\pi_2\circ g=\pi_2\circ \Delta_Y\circ f=f$ so
$f=f_1=f_2$, as desired. 

\begin{prop}
Let $X$ be a reduced scheme, $f:X\rightarrow Y$ a morphism, $Z$
a closed subscheme of $Y$, $j:Z\hookrightarrow Y$, such
that $f(X)\subseteq j(Z)$. Then $f$ factors uniquely as
$$X\stackrel{g}{\rightarrow}Z\stackrel{j}{\hookrightarrow}Y.$$
\end{prop}
\proof
First assume $X$ and $Y$ are affine, $X=\spec A$, $Y=\spec B$, 
$Z=\spec B/I$.
(Use exercise 3.11 to see that every closed subscheme $Z$ of
$Y$ is of the form $\spec B/I$.)  
Let $\phi:B\rightarrow A$ be the homomorphism which induces $f$. 
Since $f(X)\subseteq Z$, the inverse image of any prime of $A$
contains $I$. Since $A$ is reduced the intersection of all
primes of $A$ equals $\{0\}$. Thus
$$\ker(\phi)=\phi^{-1}(\{0\})=\phi^{-1}(\cap_{\mathrm{primes p}} p)\subseteq I$$
so $\phi$ factors uniquely through $B/I$. 
$$B\stackrel{j^{\#}}{\rightarrow}B/I\stackrel{g^{\#}}{\hookrightarrow}A$$
This proves the proposition when $X$ and $Y$ are affine.

Now suppose $X$ is an arbitrary reduced scheme. Cover $X$ by open affines
$X=\cup_i U_i$. For each $i$ let $g_i$ be the unique map which factors
$f|_{U_i}$ through $Z$. By uniqueness $g_i|_{U_i\cap U_j}=g_j|_{U_i\cap U_j}$,
so we can glue the $g_i$ to obtain a morphism $g:X\rightarrow Z$ 
such that $j\circ g=f$. Now suppose both $X$ and $Y$ are arbitrary. Cover $Y$ by open affines,
take their inverse images in $X$, perform the construction locally 
for each one, use uniqueness and glue. 

\noindent{\em Counterexamples.} 

(a) Let $A=k[x,y]/(x^2,xy)$, let $X=Y=\spec A$ and let $S=\spec k$. 
Then $Y$ is affine  hence seperable over $S$, but $X$ is not reduced. 
Let $f:X\rightarrow Y$ be the morphism induced by the identity homomorphism  
$\id: A\rightarrow A$. Let $g:X\rightarrow Y$ be the morphism induced
by the homomorphism $\phi: A\rightarrow A: x\mapsto 0, y\mapsto y$.    
Let $U=D(y)=\spec A_y$. Then since $A_y\cong \spec k[y,y^{-1}]$, 
the localized homomorphisms agree, $\id_y=\phi_y$. Thus 
$f|_U=g|_U$. Now $X$ is irreducible since $A$ has just one
minimal prime, namely $(x)$, so $U$ is dense in $X$. But, $f\not=g$.
since $f^{\#}=\id\not=\phi=g^{\#}$. 

(b) Let $X$ be the affine line and $Y$ the affine line
with a doubled origin both thought of as schemes over $S=\spec k$. 
Let $f_1:X\rightarrow Y$ be one of the inclusions of the affine
line in $Y$ and let $f_2:X\rightarrow Y$ be the other one.
Then $f_1$ and $f_2$ agree on $X$ minus the origin but not on $X$.  

[Reference, EGA, I.8.2.2.1.]

\begin{prob}[4.4]
Let $f:X\rightarrow Y$ be a morphism of separated schemes of finite type
over a noetherian scheme $S$. Let $Z$ be a closed subscheme of $X$ which
is proper over $S$. Show that $f(Z)$ is closed in $Y$, and that $f(Z)$
with its image subscheme structure is proper over $S$. 
\end{prob}
\proof
First we show that since $X$, $Y$ and $Z$ are of finite type 
over $S$ and $S$ is noetherian, $X$, $Y$ and $Z$ are noetherian. 
Suppose $g:X\rightarrow S$ is the map from $X$ to $S$. Cover
$S$ by finitely many $\spec A_i$, $A_i$ noetherian. Then for
each $f^{-1}(\spec A_i)=\cup_j \spec B_{ij}$, with $B_{ij}$
a finitely generated $A_i$-algebra. Since $A_i$ is noetherian
each $B_{ij}$ is noetherian (this is the Hilbert basis theorem).
Since $X=\cup_{ij}\spec B_{ij}$, $X$ is noetherian. One shows
that $Y$ and $Z$ are noetherian in exactly the same way.

Since the following diagram commutes 4.8(e) implies
$f|_Z$ is proper. \diagram 
Thus $f|_Z(Z)$ is closed in $Y$. 
(I'm assuming $Z$ is an $S$-subscheme of $X$ so that 
the diagram must commute.) 

We now have $f(Z)\hookrightarrow Y\rightarrow S$. We must
show the composition $f(Z)\rightarrow S$ is proper. By Corollary 4.8a the 
closed immersion $f(Z)\hookrightarrow Y$ is proper. 
We are given that the map $Y\rightarrow S$ is seperated
and of finite type. Since the composition of seperated morphisms
is seperated and the composition of finite type morphisms is of
finite type, the morphism $f(Z)\rightarrow S$ is seperated and
of finite type. The hard part is to show that it is universally closed.

Since the morphism $Z\rightarrow S$ is proper it is closed
and from above the morphism $Z\rightarrow f(Z)$ is closed so
the morphism $f(Z)\rightarrow S$ is closed. Let $W$ be an
any scheme over $S$. We must show that the map 
$f(Z)\times_S W\rightarrow W$ is closed. 
We have the following diagram.
\diagram \diagram
If we can show that $g_1$ is surjective we will be done. For then
if $A$ is closed subset of $f(Z)\times_S W$, 
$$g_2(A)=g_3(g_1^{-1}(A)),$$
which, since $g_3$ is closed, is also closed. (We wouldn't have 
equality in the above expression if $g_1$ weren't surjective.)

In order to establish the surjectivity of $g_1$ we prove that
the property of being a surjective morphism is preserved under
base extension. It will then follow, since $g_1$ is a base
extension of the surjective morphism $Z\rightarrow f(Z)$,
that $g_1$ is surjective.  

\begin{prop}
Let $X$ and $Y$ be schemes over $S$. Suppose $x\in X$ and $y\in Y$ 
both lie over the same point $s\in S$. Then there exists 
$\alpha\in X\times_S Y$ such that $p_X(\alpha)=x$ and
$p_Y(\alpha)=y$.
\end{prop}
\proof 
Let $g_1:\spec(k(x))\rightarrow X$ and 
$g_2:\spec(k(y))\rightarrow Y$ be the natural maps
with $g_1((0))=x$ and $g_2((0))=y$. 
Let $Z=\spec(k(x))\times_{\spec(k(s))}\spec(k(y))
=\spec(k(x)\otimes_{k(s)}k(y))$, and let $\pi_1$ 
be the projection to $\spec(k(x))$, $\pi_2$ 
the projection to $\spec(k(y))$. 
Let $g=g_1\times_S g_2$ be the product of $g_1\circ \pi_1$
with $g_2\circ \pi_2$. So $g:Z\rightarrow X\times_S Y$.
See the following diagram.
\diagram
Since $Z$ is the spectrum of the tensor product of two
fields over a common base field, $Z\not=\emptyset$ so 
there is some $z\in Z$.  By the definition of $g$, 
$g_1\circ\pi_1=p_x\circ g$ and 
$g_2\circ\pi_2=p_y\circ g$ so
$x=g_1\circ\pi_1(z)=p_x\circ g(z)$ and
$y=g_2\circ\pi_2(z)=p_y\circ g(z)$ so
we may take $\alpha=g(z)$.  

\begin{prop}
If $f:X\rightarrow Y$ is a surjective $S$-morphism then 
$f\times 1: X\times_S S' \rightarrow Y\times_S S'$ is surjective.
\end{prop}
\proof
We have the following diagram.
\diagram
Let $y'\in Y\times_S S'$. Then 
$q(y')\in Y=f(X)$ so there is $x\in X$
such that $f(x)=q(y')$. Then by the above proposition
there is some $\alpha\in X\times_S S'$ such that
$p(\alpha)=x$ and $(f\times 1)(\alpha)=y'$. Thus 
$f\times 1$ is surjective. 


\end{document}
