%%%%%%%%%%%%%%%%%%%%%%%%%%
%% Homework assignment 2
%%%%

\documentclass[12pt]{article}
\textwidth=1.2\textwidth
\textheight=1.3\textheight
\hoffset=-.7in
\voffset=-1.28in
\usepackage{amsmath}
\usepackage{amsthm}
\usepackage{amsopn}
\usepackage{amscd}

\font\bbb=msbm10 scaled \magstep 1
\font\german=eufm10 scaled \magstep 1
\font\script=rsfs10 scaled \magstep 1

\newcommand{\nd}{\not\!\!|}
\newcommand{\m}{\mathbf{m}}
\newcommand{\cross}{\times}
\newcommand{\diff}{\Omega_{B/A}}
\newcommand{\injects}{\hookrightarrow}
\newcommand{\pt}{\mbox{\rm pt}}
\newcommand{\dual}{\vee}
\newcommand{\bF}{\mathbf{F}}
\newcommand{\bZ}{\mathbf{Z}}
\newcommand{\bR}{\mathbf{R}}
\newcommand{\bQ}{\mathbf{Q}}
\newcommand{\bC}{\mathbf{C}}
\newcommand{\bA}{\mathbf{A}}
\newcommand{\bP}{\mathbf{P}}
\newcommand{\sP}{\mathcal{P}}
\newcommand{\M}{\mathcal{M}}
\newcommand{\sO}{\mathcal{O}}
\newcommand{\sox}{\sO_X}
\newcommand{\soc}{\sO_C}
\newcommand{\soy}{\sO_Y}
\newcommand{\ox}{\omega_X}
\newcommand{\so}{\sO}
%\newcommand{\sF}{\mbox{\script F}}
\newcommand{\sQ}{\mathcal{Q}}
\newcommand{\sR}{\mathcal{R}}
\newcommand{\sE}{\mathcal{E}}
\newcommand{\E}{\mathcal{E}}
\newcommand{\sF}{\mathcal{F}}
%\newcommand{\sL}{\mbox{\script L}}
\newcommand{\sL}{\mathcal{L}}
\newcommand{\sM}{\mathcal{M}}
\newcommand{\sN}{\mathcal{N}}
\newcommand{\sA}{\mathcal{A}}
\newcommand{\sD}{\mathcal{D}}
\newcommand{\sC}{\mathcal{C}}
\newcommand{\sG}{\mathcal{G}}
\newcommand{\sB}{\mathcal{B}}
\newcommand{\sK}{\mathcal{K}}
\newcommand{\sH}{\mathcal{H}}
\newcommand{\sI}{\mathcal{I}}
\newcommand{\sU}{\mbox{\german U}}
\newcommand{\gm}{\mbox{\german m}}
\newcommand{\isom}{\cong}
\newcommand{\tensor}{\otimes}
\newcommand{\into}{\rightarrow}
\newcommand{\soq}{\sO_{Q}}
\newcommand{\soxp}{\sO_{X,p}}
\newcommand{\Cone}{Q_{\text{cone}}}
\newcommand{\cech}{\v{C}ech}
\newcommand{\cH}{\text{\v{H}}}
\newcommand{\intersect}{\cap}
\newcommand{\union}{\cup}
\newcommand{\iso}{\xrightarrow{\sim}}
\newcommand{\qone}{Q_{\text{one}}}
\newcommand{\qns}{Q_{\text{ns}}}
\newcommand{\OX}{\Omega_X}

\newcommand{\F}{\sF}
\renewcommand{\P}{\bP}
\newcommand{\A}{\bA}
\newcommand{\C}{\bC}
\newcommand{\Q}{\bQ}
\newcommand{\R}{\bR}
\newcommand{\Z}{\bZ}

\newcommand{\chose}[2]{ {{#1} \choose {#2}} }
\renewcommand{\L}{\mathcal{L}}


\DeclareMathOperator{\cd}{cd}
\DeclareMathOperator{\Ob}{Ob}
\DeclareMathOperator{\Char}{char}
\DeclareMathOperator{\aut}{Aut}
\DeclareMathOperator{\End}{End}
\DeclareMathOperator{\gl}{GL}
\DeclareMathOperator{\slm}{SL}
\DeclareMathOperator{\supp}{supp}
\DeclareMathOperator{\spec}{Spec}
\DeclareMathOperator{\Spec}{Spec}
\DeclareMathOperator{\ext}{Ext}
\DeclareMathOperator{\Ext}{Ext}
\DeclareMathOperator{\tor}{Tor}
\DeclareMathOperator{\Hom}{Hom}
\DeclareMathOperator{\Aut}{Aut}
\DeclareMathOperator{\PGL}{PGL}
\DeclareMathOperator{\shom}{\mathcal{H}om}
\DeclareMathOperator{\sHom}{\mathcal{H}om}
\DeclareMathOperator{\sext}{\mathcal{E}xt}
\DeclareMathOperator{\proj}{Proj}
\DeclareMathOperator{\Pic}{Pic}
\DeclareMathOperator{\pic}{Pic}
\DeclareMathOperator{\pico}{Pic^0}
\DeclareMathOperator{\gal}{Gal}
\DeclareMathOperator{\imag}{Im}  
\DeclareMathOperator{\Id}{Id}  
\DeclareMathOperator{\Ab}{\mathbf{Ab}}
\DeclareMathOperator{\Mod}{\mathbf{Mod}}
\DeclareMathOperator{\Coh}{\mathbf{Coh}}
\DeclareMathOperator{\Qco}{\mathbf{Qco}}
\DeclareMathOperator{\hd}{hd}
\DeclareMathOperator{\depth}{depth}
\DeclareMathOperator{\trdeg}{trdeg}
\DeclareMathOperator{\rank}{rank}
\DeclareMathOperator{\Tr}{Tr}
\DeclareMathOperator{\length}{length}
\DeclareMathOperator{\Hilb}{Hilb}
\DeclareMathOperator{\Sch}{\mbox{\bfseries Sch}}
\DeclareMathOperator{\Set}{\mbox{\bfseries Set}}
\DeclareMathOperator{\Grp}{\mbox{\bfseries Grp}}
\DeclareMathOperator{\id}{id}
\DeclareMathOperator{\codim}{codim}
\DeclareMathOperator{\Var}{Var}

\theoremstyle{plain}
\newtheorem{thm}{Theorem}[section]
\newtheorem{prop}[thm]{Proposition}
\newtheorem{claim}[thm]{Claim}
\newtheorem{cor}[thm]{Corollary}
\newtheorem{fact}[thm]{Fact}
\newtheorem{lem}[thm]{Lemma}
\newtheorem{ques}[thm]{Question}
\newtheorem{conj}[thm]{Conjecture}

\theoremstyle{definition} 
\newtheorem{defn}[thm]{Definition}

\theoremstyle{remark}
\newtheorem{remark}[thm]{Remark}
\newtheorem{exercise}[thm]{Exercise}
\newtheorem{example}[thm]{Example}
\author{William A. Stein}
\title{Homework 2, MAT256B\\Chapter III, 4.8, 4.9, 5.6}
\date{April 4, 1996}
\begin{document}
\maketitle

\section{Homework}

\begin{exercise}
{\bfseries (4.8)} \quad {\em Cohomological Dimension}. Let $X$ be a neotherian separated scheme.
We define the {\em cohomological dimension} of $X$, denoted $\cd(X)$, to
be the least integer $n$ such that $H^{i}(X,\F)=0$ for all quasi-coherent
sheaves $\F$ and all $i>n$. Thus for example, Serre's theorem (3.7) says
that $\cd(X)=0$ if and only if $X$ is affine. Grothendieck's theorem (2.7)
implies that $\cd(X)\leq\dim X$. 

(a) In the definition of $\cd(X)$, show that it is sufficient to 
consider only coherent sheaves on $X$. 

(b) If $X$ is quasi-projective over a field $k$, then it is 
even sufficient to consider only locally free coherent sheaves
on $X$. 

(c) Suppose $X$ has a covering by $r+1$ open affine subsets. Use
\cech{} cohomology to show that $\cd(X)\leq r$. 

(d) If $X$ is a quasi-projective variety of dimension $r$ over a field
$k$, then $X$ can be covered by $r+1$ open affine subsets. Conclude
that $\cd(X)\leq \dim X$. 

(e) Let $Y$ be a set-theoretic complete intersection of 
codimension $r$ in $X=\P_k^n$. Show that $\cd(X-Y)\leq r-1$. 
\end{exercise}

\begin{proof}
(a) It suffices to show that if, for some $i$,
$H^i(X,\F)=0$ for all coherent sheaves $\F$,
then $H^i(X,\F)=0$ for all quasi-coherent sheaves $\F$. 
Thus suppose the $i$th cohomology of all coherent
sheaves on $X$ vanishes and let $\F$ be quasi-coherent.
Let $(\F_{\alpha})$ be the collection of coherent subsheaves
of $\F$, ordered by inclusion. Then by (II, Ex. 5.15e)
$\varinjlim\F_{\alpha}=\F$, so by (2.9)
$$H^i(X,\F)=H^i(X,\varinjlim \F_{\alpha})=\varinjlim H^i(X,\F_{\alpha})=0.$$

(b) Suppose $n$ is an integer and $H^i(X,\F)=0$ for all coherent
locally free sheaves $\F$ and integers $i>n$. We must show
$H^i(X,\F)=0$ for all coherent $\F$ and all $i>n$, then applying
(a) gives the desired result. Since $X$ is quasiprojective
there is an open immersion 
     $$i:X\hookrightarrow Y\subset \P_k^n$$
with $Y$ a closed subscheme of $\P_k^n$ and $i(X)$ open in $Y$.
By (II, Ex. 5.5c) the sheaf $\F$ on $X$ pushes forward to a coherent
sheaf on $\F'=i_{*}\F$ on $Y$.
By (II, 5.18) we may write $\F'$ as a quotient of a locally 
free coherent sheaf $\E'$ on $Y$. Letting $\R'$ be the kernel
gives an exact sequence 
$$0\into\R'\into\E'\into\F'\into 0$$
with $R'$ coherent (it's the quotient of coherent sheaves). 
Pulling back via $i$ to $X$ gives an exact sequence 
%%%%%%%%%%%%%%%%%%%%%%%%%
%% This is probably possible because $i$ is an
%% open immersion, but there are definately some
%% details to check!! 
%%%%%%%%%%%%%%%%%%%%%%%%%
$$0\into\R\into\E\into\F\into 0$$ 
of coherent sheaves on $X$ with $E$ locally free.  
The long exact sequence of cohomology shows
that for $i>n$, there is an exact sequence
$$0=H^i(X,\E)\into H^i(X,\F) \into H^{i+1}(X,\R) \into 
         H^{i+1}(X,\E)=0.$$
$H^i(X,\E)=H^{i+1}(X,\E)=0$ because we have assumed that, for $i>n$,
cohomology vanishes on locally free coherent sheaves.
Thus $H^i(X,\F)\isom H^{i+1}(X,\R)$. But if $k=\dim X$, then
Grothendieck vanishing (2.7) implies that
$H^{k+1}(X,\R)=0$ whence $H^{k}(X,\F)=0$. 
But then applying the above argument with $\F$ replaced
by $\R$ shows that $H^{k}(X,\R)=0$ which implies $H^{k-1}(X,\F)=0$
(so long as $k-1>n$). Again, apply the entire argument
with $\F$ replaced by $\R$ to see that $H^{k-1}(X,\R)=0$. 
We can continue this descent and hence show that 
$H^{i}(X,\F)=0$ for all $i>n$.

(c) By (4.5) we can compute cohomology by using the \cech{}
complex resulting from the cover $\sU$ of $X$ by $r+1$ open
affines. By definition $\sC^p=0$ for all $p>r$ since
there are no intersections of $p+1\geq r+2$ distinct open sets 
in our collection of $r+1$ open sets. The \cech{} complex is
$$\sC^0\into\sC^1\into\cdots\into\sC^r\into\sC^{r+1}=0\into 0\into 0\into \cdots.$$
Thus if $\sF$ is quasicoherent then $\cH^p(\sU,\sF)=0$ 
for any $p>r$ which implies that $\cd(X)\leq r$.  

(d) I will first present my solution in the special
case that $X$ is projective. The more general case when $X$ is 
quasi-projective is similiar, but more complicated, and will 
be presented next. 
Suppose $X\subset\P^n$ is a projective variety of dimension $r$. 
We must cover $X$ with $r+1$ open affines. Let $U$ be nonempty
open affine subset of $X$. Since $X$ is irreducible, the irreducible
components of $X-U$ all have codimension at least one in $X$. 
Now pick a hyperplane $H$ which doesn't
completely contain any irreducible component of $X-U$. We can do
this by choosing one point $P_i$ in each of the finitely
many irreducible components of $X-U$ and choosing a hyperplane
which avoids all the $P_i$. This can be done because the field
is infinite (varieties are only defined over algebraically
closed fields) so we can always choose a vector not orthogonal
to any of a finite set of vectors.
Since $X$ is closed in $\P^n$ and $\P^n-H$ is affine, 
$(\P^n-H)\intersect X$ is an open affine subset of $X$.
Because of our choice of $H$, $U\union((\P^n-H)\intersect X)$ is only 
missing codimension two closed subsets of $X$. Let $H_1=H$
and choose another hyperplane $H_2$ so it doesn't completely
contain any of the (codimension two) irreducible components
of $X-U-(\P^n-H_1)$. Then $(\P^n-H_2)\intersect X$ is open
affine and $U\union((\P^n-H_1)\intersect X)\union((\P^n-H2)\intersect X)$  
is only missing codimension three closed subsets of $X$.
Repeating this process a few more times yields 
hyperplanes $H_1,\cdots,H_r$ so that 
$$U, (\P^n-H_1)\intersect X, \ldots, (\P^n-H_r)\intersect X$$ 
form an open affine cover of $X$, as desired.

Now for the quasi-projective case. Suppose $X\subset\P^n$ is 
quasi-projective. From (I, Ex. 3.5) we know that $\P^n$ minus
a hypersurface $H$ is affine. Note that the same proof
works even if $H$ is a union of hypersurfaces. We now
proceed with the same sort of construction as in the projective
case, but we must choose $H$ more cleverly to insure
that $(\P^n-H)\intersect{}X$ is affine. Let $U$ be a nonempty  
affine open subset of $X$. As before pick a hyperplane which 
doesn't completey contain any irreducible component of $X-U$.
Since $X$ is only quasi-projective we can't conclude that
$(\P^n-H)\intersect{}X$ is affine. But we do know that
$(\P^n-H)\intersect{}\overline{X}$ is affine. Our strategy is
to add some hypersurfaces to $H$ to get a union of hypersurfaces $S$ so that 
$$(\P^n-S)\intersect{}\overline{X}=(\P^n-S)\intersect{}X.$$
But, we must be careful to add these hypersurfaces in such a
way that $((\P^n-S)\intersect{}X)\union U$ is missing only codimension
two or greater subsets of $X$. We do this as follows. For each
irreducible component $Y$ of $\overline{X}-X$ choose a hypersurface
$H'$ which completely contains $Y$ but which does not completely
contain any irreducible component of $X-U$. That this can be done
is the content of a lemma which will be proved later (just pick a point
in each irreducible component and avoid it). Let $S$
by the union of all of the $H'$ along with $H$. Then 
$\P^n-S$ is affine and so
$$(\P^n-S)\intersect{}X=(\P^n-S)\intersect{}\overline{X}$$
is affine. Furthermore, $S$ properly intersects all
irreducible components of $X-U$, so 
$((\P^n-S)\intersect{}X)\union U$ is missing only codimension
two or greater subsets of $X$. Repeating this process
as above several times yields the desired result because
after each repetition the codimension of the resulting
pieces is reduced by 1.  
\begin{lem}
If $Y$ is a projective variety and $p_1,\ldots,p_n$ is a finite
collection of points not on $Y$, 
then there exists a (possibly reducible) hypersurface $H$ containing
$Y$ but not containing any of the $p_i$. 
\end{lem}
By a possibly reducible hypersurface I mean a union of irreducible
hypersurfaces, not a hypersurface union higher codimension varieties. 
\begin{proof}
This is obviously true and I have a proof, but I think there
is probably a more algebraic proof. Note that $k$ is infinite
since we only talk about varieties over algebraically closed fields.  
Let $f_1,\cdots,f_m$ be defining equations for $Y$. Thus
$Y$ is the common zero locus of the $f_i$ and not all $f_i$ vanish
on any $p_i$. I claim that we can find a linear combination
$\sum a_i f_i$ of the $f_i$ which doesn't vanish on any $p_i$. 
Since $k$ is infinite and not all $f_i$ vanish on $p_1$, 
we can easily find $a_i$ so that $\sum a_i f_i(p_1)\neq 0$ 
and all the $a_i\neq 0$. If $\sum a_i f_i(p_2)=0$ then,
once again since $k$ is infinite, we can easily ``jiggle''
the $a_i$ so that $\sum a_i f_i(p_2)\neq 0$ and 
$\sum a_i f_i(p_1)$ is still nonzero. Repeating this same
argument for each of the finitely many points $p_i$ gives
a polynomial $f=\sum a_i f_i$ which doesn't vanish on any $p_i$. 
Of course I want to use $f$ to define our
hypersurface, but I can't because $f$ might not be homogeneous. 
Fortunately, this is easily dealt with 
by suitably multiplying the various $f_i$ by the defining equation
of a hyperplane not passing through any $p_i$, then repeating the
above argument. Now let $H$ be the hypersurface defined by $f=\sum a_i f_i$.
Then by construction $H$ contains $Y$ and $H$ doesn't
contain any $p_i$. 
\end{proof}


(e) Suppose $Y$ is a set-theoretic complete intersection of codimension
$r$ in $X=\P_k^n$. Then $Y$ is the intersection of $r$ hypersurfaces,
so we can write $Y=H_1\intersect\cdots\intersect{}H_r$ where 
each $H_i$ is a hypersurface. By (I, Ex. 3.5) $X-H_i$ is affine for
each $i$, thus 
$$X-Y=(X-H_1)\union\cdots\union(X-H_r)$$
can be covered by $r$ open affine subsets. By (c) this
implies $\cd(X-Y)\leq r-1$ which completes the proof.  
\end{proof}


\begin{exercise}
{\bfseries (4.9)} Let $X=\spec k[x_1,x_2,x_3,x_4]$ be affine four-space
over a field $k$. Let $Y_1$ be the plane $x_1=x_2=0$ and let $Y_2$
be the plane $x_3=x_4=0$. Show that $Y=Y_1\union Y_2$
is not a set-theoretic complete intersection in $X$. Therefore
the projective closure $\overline{Y}$ in $\P_k^4$ is not
a set-theoretic complete intersection.
\end{exercise}
\begin{proof}
By (Ex. 4.8e) it suffices to show that 
$H^2(X-Y,\so_{X-Y})\neq 0$.  
Suppose $Z$ is a closed subset of $X$, then by (Ex. 2.3d), 
for any $i\geq 1$, there is an exact sequence 
$$H^i(X,\sox)\into H^i(X-Z,\so_{X-Z})\into
  H_{Z}^{i+1}(X,\sox)\into H^{i+1}(X,\sox).$$
By (3.8), $H^i(X,\sox)=H^{i+1}(X,\sox)=0$ so 
$H^i(X-Z,\so_{X-Z})=H_{Z}^{i+1}(X,\sox)$.
Applying this with $Z=Y$ and $i=2$ shows that
$$H^2(X-Y,\so_{X-Y})=H_{Y}^3(X,\sox).$$ Thus
we just need to show that $H_{Y}^3(X,\sox)\neq 0$.

Mayer-Vietoris (Ex. 2.4) yields an exact sequence
\begin{align*}
H^3_{Y_1}(X,\sox)\oplus{}H^3_{Y_2}(X,\sox)\into{}H^3_Y(X,\sox)\into\\
H^4_{Y_1\intersect{}Y_2}(X,\sox)\into{}
H^4_{Y_1}(X,\sox)\oplus{}H^4_{Y_2}(X,\sox)
\end{align*}
As above, $H^3_{Y_1}(X,\sox)=H^2(X-Y_1,\so_{X-Y_1})$. 
But $X-Y_1$ is a set-theoretic complete intersection 
of codimension $2$ so $\cd(X-Y_1)\leq 1$,
whence $H^2(X-Y_1,\so_{X-Y_1})=0$. Similiarly
$$H^2(X-Y_2,\so_{X-Y_2})=H^3(X-Y_1,\so_{X-Y_1})
=H^3(X-Y_2,\so_{X-Y_2})=0.$$ Thus from the above exact
sequence we see that
$H^3_Y(X,\sox)=H^4_{Y_1\intersect{}Y_2}(X,\sox).$

Let $P=Y_1\intersect{}Y_2=\{(0,0,0,0)\}$. 
We have reduced to showing that 
$H^4_{P}(X,\sox)$ is nonzero. Since $H^4_{P}(X,\sox)=H^3(X-P,\so_{X-P})$
we can do this by a direct computation of $H^3(X-P,\so_{X-P})$ using 
\cech{} cohomology. Cover $X-P$ by the affine
open sets $U_i=\{x_i\neq{}0\}$. Then the \cech{} complex is
$$\begin{array}{l}
k[x_1,x_2,x_3,x_4,x_1^{-1}]\oplus\cdots\oplus  
k[x_1,x_2,x_3,x_4,x_4^{-1}]\xrightarrow{d_0}\\
k[x_1,x_2,x_3,x_4,x_1^{-1},x_2^{-1}]\oplus\cdots\oplus  
k[x_1,x_2,x_3,x_4,x_3^{-1},x_4^{-1}]\xrightarrow{d_1}\\
k[x_1,x_2,x_3,x_4,x_1^{-1},x_2^{-1},x_3^{-1}]\oplus\cdots\oplus  
k[x_1,x_2,x_3,x_4,x_2^{-1},x_3^{-1},x_4^{-1}]\xrightarrow{d_2}\\
k[x_1,x_2,x_3,x_4,x_1^{-1},x_2^{-1},x_3^{-1},x_4^{-1}]  
\end{array}$$
Thus 
$$H^3(X-P,\so_{X-P})
       = \{x_1^ix_2^jx_3^kx_4^{\ell}:i,j,k,\ell<0\}\neq 0.$$
\end{proof}


\begin{exercise}
{\bfseries (5.6)} {\em Curves on a Nonsingular Quadric Surface.}
Let $Q$ be the nonsingular quadric surface $xy=zw$ in $X=\P_k^3$
over a field $k$. We will consider locally principal closed
subschemes $Y$ of $Q$. These correspond to Cartier divisors on
$Q$ by (II, 6.17.1). On the other hand, we know that 
$\pic Q\isom\Z\oplus\Z$, so we can talk about the
{\em type} (a,b) of $Y$ (II, 6.16) and (II, 6.6.1). Let us denote
the invertible sheaf $\sL(Y)$ by $\so_Q(a,b)$. Thus
for any $n\in\Z$, $\so_Q(n)=\so_Q(n,n).$

[{\bf Comment!} In my solution, a subscheme $Y$ of type 
$(a,b)$ corresponds to the invertible sheaf $\soq(-a,-b)$. 
I think this is reasonable since then $\soq(-a,-b)=\sL(-Y)=\sI_Y$.  
The correspondence is not clearly stated in the problem,
but this choice works.] 

\noindent(a) Use the special case $(q,0)$ and $(0,q)$, with $q>0$, when $Y$
is a disjoint union of $q$ lines $\P^1$ in $Q$, to show:
\begin{enumerate}
\item if $|a-b|\leq 1$, then $H^1(Q,\so_Q(a,b))=0$;
\item if $a,b<0$, then $H^1(Q,\so_Q(a,b))=0$;
\item if $a\leq-2$, then $H^1(Q,\so_Q(a,0))\neq 0)$.
\end{enumerate}
{\em Solution.}
First I will prove a big lemma in which I explicitely
calculate $H^1(Q,\soq(0,-q))$ and some other things
which will come in useful later. Next I give an independent
computation of the other cohomology groups (1), (2).  
\par\begin{lem} Let $q>0$, then 
$$\dim_k H^1(Q,\soq(-q,0))=H^1(Q,\soq(0,-q))=q-1.$$
Furthermore, we know all terms in the long exact
sequence of cohomology associated with the short exact sequence
$$0\into\soq(-q,0)\into\soq\into\sO_Y\into{}0.$$
\end{lem}
\begin{proof}
We prove the lemma only for $\soq(-q,0)$, since the argument
for $\soq(0,-q)$ is exactly the same. Suppose $Y$ is the disjoint
union of $q$ lines $\P^1$ in $Q$ so $\sI_Y=\soq(-q,0)$.  
The sequence 
$$0\into\soq(-q,0)\into\soq\into\sO_Y\into{}0$$
is exact. The associated long exact sequence of cohomology is
\begin{align*}
0\into&\Gamma(Q,\soq(-q,0))\into\Gamma(Q,\soq)\into\Gamma(Q,\sO_Y)\\
\into& H^1(Q,\soq(-q,0))\into{}H^1(Q,\soq)\into{}H^1(Q,\sO_Y)\\
\into& H^2(Q,\soq(-q,0))\into{}H^2(Q,\soq)\into{}H^2(Q,\sO_Y)\into 0
\end{align*}
We can compute all of the terms in this long exact sequence. For
the purposes at hand it suffices to view the summands as $k$-vector
spaces so we systematically do this throughout. 
Since $\soq(-q,0)=\sI_Y$ is the ideal sheaf of $Y$, 
its global sections must vanish on $Y$. But $\sI_Y$ is a subsheaf
of $\sO_Q$ whose global sections are the constants. Since the
only constant which vanish on $Y$ is $0$, $\Gamma(Q,\soq(-q,0))=0$.
By (I, 3.4), $\Gamma(Q,\soq)=k$. Since $Y$ is the disjoint union
of $q$ copies of $\P^1$ and each copy has global sections $k$,
$\Gamma(Q,\sO_Y)=k^{\oplus{}q}$. Since $Q$ is a complete intersection of
dimension 2, (Ex. 5.5 b) implies $H^1(Q,\soq)=0$. 
Because $Y$ is isomorphic to several copies of $\P^1$, 
the general result (proved in class, but not in the book) 
that $H_{*}^n(\sO_{\P^n})=\{\sum a_{I}X_{I}:\text{entries in $I$ negative}\}$ 
implies $H^1(Q,\sO_Y)=H^1(Y,\sO_Y)=0$. Since $Q$ is a hypersurface
of degree $2$ in $\P^3$, (I, Ex. 7.2(c)) implies $p_a(Q)=0$. Thus
by (Ex. 5.5c) we see that $H^2(Q,\soq)=0$. Putting together the above
facts and some basic properties of exact sequences show that
$H^1(Q,\soq(-q,0))=k^{\oplus(q-1)}$, $H^2(Q,\soq(-q,0))=0$
and $H^2(Q,\so_Y)=0$. Our long exact sequence is now 
\begin{align*}
0\into&\Gamma(Q,\soq(-q,0))=0\into\Gamma(Q,\soq)=k\into\Gamma(Q,\sO_Y)=k^{\oplus{}q}\\
\into& H^1(Q,\soq(-q,0))=k^{\oplus(q-1)}\into{}H^1(Q,\soq)=0\into{}H^1(Q,\sO_Y)=0\\
\into& H^2(Q,\soq(-q,0))=0\into{}H^2(Q,\soq)=0\into{}H^2(Q,\sO_Y)=0\into 0
\end{align*}
\end{proof}

Number (3) now follows immediately from the lemma because 
$$H^1(Q,\soq(a,0))=k^{\oplus(-a-1)}\neq 0$$ for $a\leq -2.$

Now we compute (1) and (2). Let $a$ be an arbitrary integer.
First we show that $\soq(a,a)=0$. We have an exact sequence
$$0\into\sO_{\P^3}(-2)\into\sO_{\P^3}\into\sO_{Q}\into{}0$$
where the first map is multiplication by $xy-zw$. Twisting by $a$ 
gives an exact sequence
$$0\into\sO_{\P^3}(-2+a)\into\sO_{\P^3}(a)\into\sO_{Q}(a)\into{}0.$$
The long exact sequence of cohomology yields an exact sequence
$$\cdots\into H^1(\sO_{\P^3}(a))\into H^1(\sO_Q(a))\into
        H^2(\sO_{\P^3}(-2+a))\into\cdots$$
But from the explicit computations of projective space (5.1)
it follows that $H^1(\sO_{\P^3}(a))=0$ and $H^2(\sO_{\P^3}(-2+a))=0$
from which we conclude that $H^1(\sO_Q(a))=0$.

Next we show that $\soq(a-1,a)=0$. Let $Y$ be a single copy 
of $\P^1$ sitting in $Q$ so that $Y$ has type $(1,0)$. Then
we have an exact sequence
$$0\into\sI_Y\into\soq\into\sO_Y\into 0.$$
But $\sI_Y=\soq(-1,0)$ so this becomes
$$0\into\soq(-1,0)\into\soq\into\sO_Y\into 0.$$
Now twisting by $a$ yields the exact sequence
$$0\into\soq(a-1,a)\into\soq(a)\into\sO_Y(a)\into 0.$$
The long exact sequence of cohomology gives an exact sequence
$$\cdots\into\Gamma(\soq(a))\into\Gamma(\sO_Y(a))\into
       H^1(\soq(a-1,a))\into H^1(\soq(a))\into\cdots$$
We just showed that $H^1(\soq(a))=0$, so to see that
$H^1(\soq(a-1,a))=0$ it suffices to note that the
map $\Gamma(\soq(a))\into\Gamma(\sO_Y(a))$ is surjective.
This can be seen by writing $Q=\proj(k[x,y,z,w]/(xy-zw))$
and (w.l.o.g.) $Y=\proj(k[x,y,z,w]/(xy-zw,x,z))$ and
noting that the degree $a$ part of $k[x,y,z,w]/(xy-zw)$
surjects onto the degree $a$ part of $k[x,y,z,w]/(xy-zw,x,z)$. 
Thus $H^1(\soq(a-1,a))=0$ and exactly the same argument
shows $H^1(\soq(a,a-1))=0$. This gives (1). 

For (2) it suffices to show that for $a>0$, 
$$H^1(\soq(-a,-a-n))=H^1(\soq(-a-n,-a))=0$$
for all $n>0$. Thus let $n>0$ and suppose $Y$ is a disjoint union
of $n$ copies of $\P^1$ in such a way that $\sI_Y=\soq(0,-n)$. 
Then we have an exact sequence
$$0\into\soq(0,-n)\into\soq\into\sO_Y\into 0.$$
Twisting by $-a$ yields the exact sequence
$$0\into\soq(-a,-a-n)\into\soq(-a)\into\sO_Y(-a)\into 0.$$
The long exact sequence of cohomology then gives an exact sequence
$$\cdots \into\Gamma(\sO_Y(-a))\into H^1(\soq(-a,-a-n))
\into H^1(\soq(-a))\into \cdots$$
As everyone knows, since $Y$ is just several copies of $\P^1$ and $-a<0$,
$\Gamma(\sO_Y(-a))=0$. Because of our computations above,
$H^1(\soq(-a))=0$. Thus $H^1(\soq(-a,-a-n))=0$, as desired. 
Showing that $H^1(\soq(-a-n,-a))=0$ is exactly the same. 

\noindent (b) Now use these results to show:
\begin{enumerate}
\item If $Y$ is a locally principal closed subscheme of
type $(a,b)$ with $a,b>0$, then $Y$ is connected.
\begin{proof}
Computing the long exact sequence associated to the short
exact sequence
$$0\into\sI_Y\into\soq\into\sO_Y\into{}0$$
gives the exact sequence
$$0\into\Gamma(Q,\sI_Y)\into\Gamma(Q,\soq)\into\Gamma(Q,\sO_Y)
\into H^1(Q,\sI_Y)\into\cdots$$
But, $\Gamma(\sI_Y)=0$, $\Gamma(Q,\soq)=k$, and
by (a)2 above $H^1(Q,\sI_Y)=H^1(Q,\soq(-a,-b))=0$. Thus we
have an exact sequence
$$0\into{}0\into{}k\into\Gamma(\so_Y)\into{}0\into\cdots$$
from which we conclude that $\Gamma(\so_Y)=k$ which implies
$Y$ is connected.  
\end{proof}

\item now assume $k$ is algebraically closed. Then for any
$a,b>0$, there exists an irreducible nonsingular curve
$Y$ of type $(a,b)$. Use (II, 7.6.2) and (II, 8.18).
\begin{proof}
Given $(a,b)$, (II, 7.6.2) gives a closed immersion
$$Q=\P^1\times\P^1\into\P^a\times\P^b\into\P^n$$
which corresponds to the invertible sheaf $\soq(-a,-b)$
of type $(a,b)$. By Bertini's theorem (II, 8.18) there
is a hyperplane $H$ in $\P^n$ such that the hyperplane section
of the $(a,b)$ embedding of $Q$ in $\P^n$ is nonsingular. 
Pull this hyperplane section back to a nonsingular curve $Y$ of
type $(a,b)$ on $Q$ in $\P^3$. By the previous problem, $Y$ is
connected. Since $Y$ comes from a hyperplane section this implies
$Y$ is irreducible (see the remark in the statement of Bertini's
theorem).  
\end{proof}

\item an irreducible nonsingular curve $Y$ of type $(a,b)$,
$a,b>0$ on $Q$ is projectively normal (II, Ex. 5.14) if 
and only if $|a-b|\leq 1$. In particular, this gives lots
of examples of nonsingular, but not projectively normal curves
in $\P^3$. The simplest is the one of type $(1,3)$ which is
just the rational quartic curve (I, Ex. 3.18). 
\begin{proof}
Let $Y$ be an irreducible nonsingular curve of type $(a,b)$. 
The criterion we apply comes from (II, Ex 5.14d) which asserts
that the maps
$$\Gamma(\P^3,\sO_{\P^3}(n))\into\Gamma(Y,\sO_Y(n))$$
are surjective for all $n\geq 0$ if and only if
$Y$ is projectively normal. To determine when this occurs
we have to replace $\Gamma(\P^3,\sO_{\P^3}(n))$ with 
$\Gamma(Q,\soq(n))$. It is easy to see that the above criterion 
implies we can make this replacement if $Q$ is projectively normal. 
Since $Q\isom\P^1\times\P^1$ is locally isomorphic to 
$\A^1\times\A^1\isom\A^2$ which is normal, we see that $Q$
is normal. Then since $Q$ is a complete intersection which is
normal, (II, 8.4b) implies $Q$ is projectively normal. 

Consider the exact sequence
$$0\into\sI_Y\into\soq\into\sO_Y.$$ 
Twisting by $n$ gives an exact sequence
$$0\into\sI_Y(n)\into\soq(n)\into\sO_Y(n).$$
Taking cohomology yields the exact sequence
$$\cdots\into\Gamma(Q,\soq(n))\into\Gamma(Q,\sO_Y(n))\into
              H^1(Q,\sI_Y(n))\into\cdots$$
Thus $Y$ is projectively normal precisely if
$H^1(Q,\sI_Y(n))=0$ for all $n\geq 0$. When can 
this happen? We apply our computations from part (a).
Since $\soq(n)=\soq(n,n)$,
$$\sI_Y(n)=\soq(-a,-b)(n)=\soq(-a,-b)\tensor_{\soq}\soq(n,n)=\soq(n-a,n-b)$$ 
If $|a-b|\leq 1$ then $|(n-a)-(n-b)|\leq 1$ for all $n$ so
$$H^1(Q,\soq(-a,-b)(n))=0$$ for all $n$ which implies $Y$ is
projectively normal. On the other hand, if $|a-b|>1$ let
$n$ be the minimum of $a$ and $b$, without loss assume $b$ is
the minimum, so $n=b$. Then from (a) we see that
$$\soq(-a,-b)(n)=\soq(-a,-b)(b)=\soq(-a+b,0)\neq 0$$
since $-a+b\leq -2$.  
\end{proof}
\end{enumerate}


(c) If $Y$ is a locally principal subscheme of type $(a,b)$
in $Q$, show that $p_a(Y)=ab-a-b+1.$ [Hint: Calculate the
Hilbert polynomials of suitable sheaves, and again use the
special case (q,0) which is a disjoint union of $q$ copies
of $\P^1$.]
\begin{proof}
The sequence 
$$0\into\soq(-a,-b)\into\soq\into\sO_Y\into{}0$$
is exact so 
$$\chi(\so_Y)=\chi(\soq)-\chi(\soq(-a,-b))=1-\chi(\soq(-a,-b)).$$
Thus 
$$p_a(Y)=1-\chi(\so_Y)=\chi(\soq(-a,-b)).$$
The problem is thus reduced to computing $\chi(\soq(-a,-b))$.

Assume first that $a,b<0$. To compute $\chi(\soq(-a,-b))$
assume $Y=Y_1\union Y_2$ where
$\sI_{Y_1}=\soq(-a,0)$ and $\sI_{Y_2}=\soq(0,-b)$. Thus
we could take $Y_1$ to be $a$ copies of $\P^1$ in one family of lines 
and $Y_2$ to be $b$ copies of $\P^1$ in the other family. 
Tensoring the exact sequence
$$0\into\sI_{Y_1}\into\soq\into\so_{Y_1}\into 0$$
by the flat module $\sI_{Y_2}$ yields an exact sequence
$$0\into\sI_{Y_1}\tensor\sI_{Y_2}\into\sI_{Y_2}\into\sO_{Y_1}\tensor\sI_{Y_2}$$
[Note: I use the fact that $\sI_{Y_2}$ is flat. This follows
from a proposition in section 9 which we haven't yet reached, 
but I'm going to use it anyways. Since $Y_2$ is locally principal,
$\sI_{Y_2}$ is generated locally by a single element and since $Q$ is
a variety it is integral. Thus $\sI_{Y_2}$ is locally free so
by (9.2) $\sI_{Y_2}$ is flat.] 
This exact sequence can also be written as
$$0\into\soq(-a,-b)\into\soq(0,-b)\into\sO_Y\tensor\soq(0,-b)\into{}0.$$
The associated long exact sequence of cohomology is
\begin{align*}
0\into&\Gamma(Q,\soq(-a,-b))\into\Gamma(Q,\soq(0,-b))\into
                \Gamma(Q,\sO_{Y_1}\tensor\soq(0,-b))\\
\into&H^1(Q,\soq(-a,-b))\into H^1(Q,\soq(0,-b))\into H^1(Q,\sO_{Y_1}\tensor\soq(0,-b))\\
\into&H^2(Q,\soq(-a,-b))\into H^2(Q,\soq(0,-b))\into H^2(Q,\sO_{Y_1}\tensor\soq(0,-b))\into 0
\end{align*}
The first three groups of global sections are $0$. Since $a,b<0$, 
(a) implies $H^1(Q,\soq(-a,-b))=0$. From the lemma we know
that $H^1(Q,\soq(0,-b))=k^{\oplus(b-1)}$. Also by the lemma
we know that $H^2(Q,\soq(0,-b))=0$. Since
$\so_{Y_1}\tensor\soq(0,-b)$ is isomorphic
to the ideal sheaf of $b-1$ points in each line
of $Y_1$, a similiar proof as that used in the
lemma shows that $$H^1(Q,\sO_Y\tensor\soq(0,-b))=k^{\oplus{}a(b-1)}.$$
Plugging all of this information back in yields the exact sequence
\begin{align*}
0\into&\Gamma(Q,\soq(-a,-b))=0\into\Gamma(Q,\soq(0,-b))=0\into
                \Gamma(Q,\sO_{Y_1}\tensor\soq(0,-b))=0\\
\into&H^1(Q,\soq(-a,-b))=0\into H^1(Q,\soq(0,-b))=k^{\oplus(b-1)}\\
          &\hspace{1in}\into{}H^1(Q,\sO_{Y_1}\tensor\soq(0,-b))=k^{\oplus{}a(b-1)}\\
\into&H^2(Q,\soq(-a,-b))\into H^2(Q,\soq(0,-b))=0\\ 
          &\hspace{1in}\into{}H^2(Q,\sO_{Y_1}\tensor\soq(0,-b))=0\into 0
\end{align*}
From this we conclude that 
$$\chi(\soq(-a,-b))=0+0+h^2(Q,\soq(-a,-b))=a(b-1)-(b-1)=ab-a-b+1$$
which is the desired result. 
    
Now we deal with the remaining case, when $Y$ is $a$ disjoint copies
of $\P^1$. We have 
$$p_a(Y)=1-\chi(\sO_Y)=1-\chi(\sO_{\P^1}^{\oplus a})
                      =1-a\chi(\sO_{\P^1})=1-a$$ 
which completes the proof.
\end{proof}
\end{exercise}
\end{document}
