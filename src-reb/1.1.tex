\def\Z{{\bf Z}}
{\bf Hartshorne, Chapter 1}
Answers to exercises. \hfill REB 1994
\item{1.1a} $k[x,y]/(y-x^2)$ is identical with its subring $k[x]$. 
\item{1.1b} $A(\Z)=k[x,1/x]$ which contains an invertible element not
in $k$ and is therefore not a polynomial ring over $k$. 
\item{1.1c} Any nonsingular 
conic in $P^2$ can be reduced to the form $xy+yz+zx=0$
and this curve is isomorphic to $P^1$. (Proof: choose any 3 points
on the conic, and choose coordinates so that these points 
are $(1:0:0), (0:1:0), (0,0,1)$; this means the conic must
have the equation $cxy+ayz+bzx=0$, with $a,b,c$ all nonzero
(otherwise the conic is singular). Then multiplying
$x,y,z$ by $a,b,c$ shows that the conic has equation $xy+yz+zx=0$.
Hence all nonsingular conics are isomorphic to this one, 
and as it is easy to find one isomorphic to $P^1$ they all are.) 
Therefore (regular function on a conic) = (regular functions
on the conic $xy+yz+zx=0$ - some hyperplane) = 
(regular functions on $P^1$ - 1 or 2 points) = $A(Y)$ or $A(z)$. 
The ring is $A(Y)$ if and only if the conic $ax^2+bxy+cy^2+$(terms of degree
$< 2$)
intersects the line at infinity
in exactly one point, which happens if and only if $b^2=4ac$.  
\item{1.2} $Y$ is isomorphic to $A^1$ and is therefore an affine variety of
dimension 1, and $A(Y)=k[x]$. $I(Y) $ is generated by $Y-X^2,Z-X^3$. 
\item{1.3} $xy=x$, so $x=0$ or $z=1$. $x^2=yz$, so $x=0,y=0$ or
$x=0,z=0$, or $z=1,x^2=y$. Therefore $Y$ is the union of 2 lines
and a parabola. The prime ideals are generated by $x,y$ or $x,z$
or $z-1,x^2-y$. 
\item {1.4} The line $x=y$ is closed in $A^2$ but not in $A^1\times A^1$
(at least if $k$ is infinite).
\item{1.5} $B$ is a finitely generated algebra over $k$ and has no nilpotents.
If $x_1,...x_n$ is a set of generators for $B$ then $B=k[x_1,\ldots,x_n]/I$
for some ideal $I$, and $\sqrt I = I$ as $B$ has no nilpotents.
Hence $I(V(I))=I$ by the nullstellensatz, so that $B$ is the coordinate
ring of $V(I)$ in $A^n$. 
\item{1.6} Put $U\subset X$, $U$ open, $X$ irreducible.
Then $X=(X-U)\cup\bar U$, so $\bar U=X$, so $U$ is dense in $X$. 
If $U\subset C_1\cup C_2$, then $X=\bar U= \bar C_1\cup \bar C_2=
C_1\cup C_2$, so $C_1$ or $C_2$ contains $U$, so $U$ is irreducible.
\item {1.7a} (i) is equivalent to (iii) by taking complements. 
(ii) implies (iv) is trivial. (i) implies (ii) because if some set contains
no smallest closed subset then we can choose an infinite descending
chain $C_1\supset C_2\supset\cdots$ using Zorn's lemma. The
proof that (iii) is equivalent to (iv) is similar.
\item{1.7b} If $U$ is any open cover of $X$, apply (a)(iv) to 
the unions of the finite subsets of $U$. 
\item{1.7c} Follows from (a)(iv).
\item {1.7d} $X$ Noetherian and Hausdorff implies $X$ Hausdorff
and every subset Noetherian implies $X$ Hausdorff and every subset compact
implies $X$ compact and every subset closed implies $X$ compact and discrete
implies $X$ finite and discrete. 
\item {1.8} Let $H$ have ideal $(f)$. As $Y$ is not contained in $H$, 
$f$ is neither a unit nor a zero divisor in the coordinate ring $B$ of
$Y$.  Therefore by 1.11A every minimal prime $P$ containing $f$ has
height 1. By 1.8A $\dim(B/P)=r-1$. If $X$ is an irreducible component
of $Y\cup H$ then the ideal of $X$ is a minimal prime ideal $P$ of $B$
containing $f$ and the coordinate ring of $X$ is $B/P$.
\item{1.9} The dimension of any component of $Z(a)$ = transcendence degree
of its function field. This function field contains $x_1,\ldots, x_n$ and
the algebraic relations between these are a consequence of the $r$ 
generators of $a$. Therefore the dimension of any component 
is at least $n-$number of generators of $a \ge n-r$. 
\item {1.10a} If $Y_0\subset Y_1 \subset \cdots \subset Y_n$ is a chain of irreducible
closed subsets of $Y$, then $\bar Y_0\subset \bar Y_1 \subset \cdots
\subset \bar Y_n$ is a chain of irreducible closed subsets of $X$. 
\item {1.10b} By (a), $\dim(X)\ge \sup \dim(U_i)$. If $X_0\subset\cdots X_n$
is a sequence of irreducible closed subsets of $X$ with $X_0$ a point, 
choose some set $U$ in the cover with $X_0\in U$. Then by 1.6 $X_i\cap U$ is
irreducible and dense in $X_i$ and therefore not contained in $X_{i-1}$. 
Hence $X_0\cap U\subset X_1\cap U\subset\cdots \subset X_n\cap U$
is a sequence of closed strictly increasing irreducible subsets of $U$, 
so $\dim(X)\le \dim U\le \sup \dim U_i$. 
\item {1.10c} $X=\{u,v\}$ (a 2 point set) with open sets 
$\emptyset,\{u\}=U, X$. 
\item{1.10d} If $Y_0\subset\cdots\subset Y_n$ is a chain of
closed irreducible subsets of $Y$ and $Y\ne X$, then we can 
add $X$ to the end of this chain to see that $\dim(X)\ge \dim(Y)+1$
so either $\dim(X)=\infty$ or $\dim (X)>\dim(Y)$. 
\item {1.10e} The set of positive integers, closed sets
those of the form $\{1,2,3,\ldots,n\}$. 
\item {1.11} $t\rightarrow (t^3,t^4,t^5)$ is a homeomorphism from
$A^1$ to $Y$, so $\dim(Y)=1$, so $P$ has height 2. No element of
the ideal of $P$ has homogeneous components of degree 0 or 1, 
and the possible homogeneous components of degree 2 form 
a vector space of dimension 3, so $P$ needs at least 
3 generators. ($P$ is generated by $x^2y-z^2, zx-y^2,x^3-zy$.)
\item{1.12} $f(x,y)= y^4+y^2 +x^2(x-1)^2$.
\end
