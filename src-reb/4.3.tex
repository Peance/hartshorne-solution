%included selected solutions to some problems in IV 3
%run latex and print out via its .dvi file
\documentstyle[amssymb]{article}
\input amssym.def
\input amssym
\begin{document}

\newcommand{\pa}{{\Bbb P}^1}
\newcommand{\pp}{{\Bbb P}^2}
\newcommand{\ppp}{{\Bbb P}^n}

{\bf Solution to Math256a section IV.3 (H.Zhu, 1994)}

3.1) One direction follows easily from 3.3.4.  We show that $D$ is not
very ample when $deg(D)<5$. Now suppose $D$ is very ample, then
$l(D)=l(D-P-Q)+2\geq 2$. Futhermore, if $l(D)=2$, $dim|D|=1$, thus
$|D|$ defines an isomorphism from $X$ to $\pa$, which is obsurd. Thus
we have $l(D)>2$.

If $deg(D)\leq 1$, Since $l(D)\not =0$, we may apply Ex. 1.5.,
$l(D)\leq deg(D)+1\leq 2$, thus $D$ is not very ample.

If $deg(D)=2$, $l(D)=l(K-D)+1$. Since $D\not =0$, $l(K-D)<l(K)=2$. Thus
$l(D)\leq 2$. Contradiction.

If $deg(D)=3$, then $l(K-D)=0$. So $l(D)=2$. Contradiction.

If $deg(D)=4$, then $l(D)=3$. By 3.2, we know that $D$ is base point
free. Thus $|D|$ defines a morphism from $X$ to $\pp$. But this is
impossible since any plane curve has genus $(d-1)(d-2)/2$, which is
never 2. Contradiction.

We conclude that $deg(D)\geq 5$.

3.2) (a) From I, Ex.7.2, $g(X)=3$. It results in $l(K)=3$ and
$deg(K)=4$. Denote $D=:X.L$. Recall Bezout's theorem from I, 7. so
$deg(D)=4$. Now claim that $l(D)\geq 3$. Since the line $L$ on $X$ is
determined exactly by two points (not necessary distinct) so $dim
|L|=2$, i.e. $l(D)=3$. (This may be rigorously proved by considering
the possible linearly independent sets.)  Then
$l(K-D)=l(D)+g-deg(D)-1=1$. But $deg(K-D)=0$ and $l(K-D)=deg(K-D)+1$,
thus $K=D$ by Ex.1.5.

(b) Since $D$ is an effective divisor of degree $2$, $D=P_1+P_2$ for
some two points on $X$ (not necessary distinct). Suppose there is an
effective divisor $Q_1+Q_2$ such that $P_1+P_2\sim Q_1+Q_2$. Since the
line passing thru $P_1$ and $P_2$ intersects $X$ at two other points
$P_3$ and $P_4$. By (a) we have $K=P_1+P_2+P_3+P_4$, so $Q_1, Q_2,
P_3, P_4$ is collinear. Hence $Q_1, Q_2$ coincide with $P_1, P_2$.
Thus $dim|D|=0$.

(c) From Ex. 1.7.(a), $dim|K|=1$. But we may pick an effective
canonical divisor $K$ such that $dim|K|=0$ by (b). Thus $X$ can not be
a hyperelliptic curve. 

3.3) It is clear that the second statement follows from the first one
since $K$ is not very ample on a hyperelliptic curve. (Cf. 5.2.)  By
II.Ex.8.4, $\omega_X\cong {\cal O}(\sum{d_i}-n-1)$. Since the
dimension of the global section of this invertible sheaf equals $g\geq
2$, $\omega_X$ has to be very ample. (Otherwise it has no global
sections.)  This is equivalent to saying that the canonical divisor
$K$ is very ample.

We showed in Ex.1.7 that any curve of genus $2$ has to be a
hyperelliptic curve, and its canonical divisor is not very ample.
Thus it can not be a complete intersection in $\ppp$.

3.4) (a) Denote $\theta$ as the corresponding ring homomorphism. $deg
(\theta)=d$. We know that the image of the $d$-uple embedding is
$Z(ker(\theta))$. We may check that $ker(\theta)$ is generated by
$x_{i+1}^2-{x_i}{x_{i+2}}$, for $i=0,\cdots, d-2$, and
$x_0x_d-x_1x_{d-1}$.

(b) If $i$ is the close immersion, denote $i^*({\cal O}(1))$ as
$D$. Because $dim |D|=n$ and $deg(D)=d$, we have $l(D)=n+1\leq
deg(D)+1=d+1\leq n+1$. Therefore $n=d$ and $g=0$ by
Ex..1.5. Consequently, $X\cong \pa$. Since $X$ does not lie in ${\Bbb
P}^{n-1}$, the natural map ${\Gamma}(\ppp,{\cal O}(1))\rightarrow
{\Gamma}(X,i^*({\cal O}(1))$ is injective. Thinking $X$ as $\pa$, $D$
corresponds to a $(n+1)$-dimension subspace $V\in {\Gamma}(\pa,{\cal
O}(n))$ hence they are equal since the later has dimension
$n+1$. Therefore, $X$ is indeed a rational normal curve.  (see
II. 7.8.1).

(c) It is clear this curve $X$ can not be in $\pa$. From (b), $X$ is a
plane curve of degree $2$.  

(d) Suppose $X$ is not a plane cubic curve, we apply (b), have
$X\subseteq {\Bbb P}^3\backslash {\Bbb P}^2$, thus it is a rational
normal curve of degree $3$, which is indeed a twisted cubic.

3.6) (a) When $n\geq 4$, Ex.3.4 (b) implies that $X$ is a rational
normal curve. If $X$ is a plane curve, $g(X)=(d-1)(d-2)/2=3$.
Otherwise, $X\subseteq {\Bbb P}^3\backslash {\Bbb P}^2$, we claim that
$g=0,1$. Suppose $g=2$, then Ex.3.1 shows that any divisor of degree
$4$ is not very ample, that is $X$ can not be embedded to ${\Bbb
P}^3$, which is absurd. If $g=0$, then it is a rational quartic curve
by II,7.8.6. $g$ can not be $3$ since $X$ is not a plane curve.  Thus
$g$ has to be 1.

3.7) Suppose $C$ is a nonsingular curve which projects to the given
curve $X$. We prove that $deg(C)=4$ which will soon lead a
contradiction with assertions in Ex.3.6. To prove our first claim,
we carefully choose a suitable hyperplane $H$ passing the projection
point to cut $C$ which intersects with $\pp$ by a line $L$ such that
there is a 1-1 map from $C.H$ to $X.L$. We conclude that $deg(C)=4$ by
recalling Bezout's Theorem. 

Since $C$ has a node, it can not lie in case (1) or (2) in Ex.3.6. By
Hurwitz's theorem, $g(C)\geq g(\tilde{X})=3-1$ from 3.11.1, thus
$g(X)\not =1$. Contradiction with Ex.3.6. Thus such $C$ does not
exist.

3.8) (a) By a simple calculation, the tangent vector is $(1,0,0)$ at
each point. Pick an point $P=(x_0,y_0,z_0)$ on $X$, its tangent line
is given by the intersection of two hyperplanes: $y=y_0$ and $z=z_0$.
Writen in homogeneous polynomial, $y=y_0w$ and $z=z_0w$. Thus all
tangent lines pass through the point at infinity $(1:0:0:0)$. There is
one strange point on this curve.  

(b) Note that when $char(k)=0$, $X$ has finitely many singular points.
By choosing a proper projection, we may still project $X$ in ${\Bbb
P}^3$. Suppose $P$ is a strange point on $X$. Choose an affine cover
such that $P$ is the infinity point on $x$-axis, and other relevant
conditions in the proof of Theorem 3.9. The resulted morphism is
ramified at all but finitely many points on $X$. The image is thus a
point otherwise the map is inseparable which is not the case over a
field of $char$ 0. Hence $X$ is the line $\pa$.

3.9) Three points are collinear iff there is a multisecant line passing
through them. A hyperpalne in ${\Bbb P}^3$ intersects $X$ at exactly
$d$ points iff the hyperplane does not pass any tangent lines of $X$. 
Prop 3.5 showed that $dim(Tan(X))\leq 2$. Also it is not hard to show
that the dimension of the space of multisecant lines of $X$ has
dimension $\leq 1$. Hence the union of these two spaces is a proper
closed subspace of ${{\Bbb P}^3}^*$ which is of dimension
$3$. Therefore almost all hyperplanes intersect $X$ in exactly $d$
points. 

\end{document}