\def\Z{{\bf Z}}
{\bf Hartshorne, Chapter 1.4}
Answers to exercises. \hfill REB 1994
\item{4.1} If $f=g$ on $U\cap V$, then the function which is $f$ on $U$ 
and $g$ on $V$ is clearly regular. Therefore the union of all open
sets on which $f$ is represented by a regular function is the largest
open set on which $f$ is regular.
\item{4.2} A map is regular if and only if it is regular in a neighborhood
of each point, so the conclusion follows as in 4.1.
\item{4.3a} $f=x_1/x_0$ is defined in the set where $x_0\ne 0$. 
This set is isomorphic to $A^2$, and $f$ is just projection to
the first coordinate. 
\item{4.3b} $\phi$ is defined everywhere except the point $(1:0:0)$. 
\item{4.4a} See exercise 3.1c.
\item{4.4b} The maps taking $t\in A^1$ to $(t^3,t^2)$ 
and $(x,y)$ to $y/x$ (for $x\ne 0$) are inverse birational
isomorphisms from the cuspidal curve to $A^1$. 
\item{4.4c} The projection maps $(x:y:z)$ to $(x:y)$ if $(x:y:z)\ne (0:0:1)$.
The inverse map from $P^1$ to $Y$ takes $(x:y)$ to
$((y^2-x^2)x:(y^2-x^2)y:x^3)$ for $(x:y)\ne (1:\pm 1)$. 
\item{4.5} The subvariety of $Q$  given by $w\ne 0$ is isomorphic
to $A^2$ by $(w:x:y:z)\rightarrow (x/w,y/w)$, $(x,y)\rightarrow (1:x:y:xy)$.
Thererfore $Q$ is birational to $Q-\{w=0\}$, which is isomorphic
to $A^2$, which is birational to $P^2$. $Q$ is isomorphic to 
$P^1\times P^1$, which is not isomorphic to $P^2$ as it contains
2 closed 1-dimensional subvarieties that do not intersect. 
\item{4.6ab} Put $U=V=\{(x:y:z)|xyz\ne 0\}$. $\Phi$ clearly maps
$U$ to $V$, and $\phi^2$ maps $(x:y:z)$ to $(ax:xy:az)=(x:y:z)$
where $a=xyz$, so $\phi^2$ is the identity map. 
\item{4.6c} $\phi=\phi^{-1}$  is defined everywhere 
on $P^2$ except at the points $(1:0:0)$, $(0:1:0)$, $(0:0:1)$. 
Remark: The group of birational transformations
of $P^2$ is generated by quadratic transformations
(or by one quadratic transformation and $PGL_3(k)$)
and very little about it seems to be known
beyond the fact that it is very large.
\item{4.7} We can assume that $X$ and $Y$ are closed subsets of $A^n$,
and $P=Q=0$. If $f$ is a homomorphism from $O_{Q,Y}$ to
$O_{P,X}$ then define a map $g$ from an open subset 
of $X$ to $Y$ by 
$$g(x_1,\ldots, x_n)=(f(y_1)(x_1,\ldots,x_n), f(y_2)(x_1,\ldots),\ldots)$$
where $y_i$ is the $i$'th coordinate function on $A^n$. 
This is defined on the open set where all the $f(y_i)$'s are defined.
Likewise we can define a similar map from an open set of
$Y$ to $X$, and the composition of these two maps is the identity wherever
it is defined. Therefore there is an isomorphism from 
an open set of $X$ to an open set of $Y$ taking $P$ to $Q$. 
\item{4.8a} Clearly the cardinality of $P^n$ is at most
$(n+1)card(k)^n$ which is the cardinality of $k$. To prove the other
inequality we can assume that $X$ is contained in $A^n$.
If the possible values of any coordinate $x_1,\ldots ,x_n$ are finite,
the $X$ consists of a finite number of points, so we can assume that
 one coordinate, say $x_1$, takes on an infinite number of values. 
By elimination theory the condition for 
a point with a given value of $x_1$ to exist on $X$ is given
by a finite number of equalities and inequalities in $x_1$. 
Therefore the possible values of $x_1$ are either a finite set
or the complement of a finite set in $k$. But we know the number of
possible values of $x_1$ is infinite, so the number of values
is the cardinality of $k$ minus a finite number, which is the
cardinality of $k$. 
\item{4.8b} Any two curves have the same cardinality and 
the finite complement topology, and so are homeomorphic.
\item{4.9} We can assume that $X$ is affine and is contained in $A^n$,
the set of points in $P^n$ with first $x_0\ne 0$. The field of fractions
$k(X)$ is generated by $x_1,\ldots, x_n$, so we can assume that
$x_1,\ldots, x_r$ is a separating transcendence basis for $k(X)/k$
by 4.7A and 4.8A, and $k(X)$ is generated by
$a_{r+1}x_{r+1}+\cdots+a_nx_n$ for some $a_i$'s in $k$, by 4.6A. 
As $r\le n-2$ we can find a form $b_{r+1}x_{r+1}+\cdots+b_nx_n$
not proportional to $a_{r+1}x_{r+1}+\cdots+a_nx_n$. Choose any point at 
infinity not in this plane or in $\bar X$. Then the projection 
from this point to the plane maps $k(hyperplane)$ onto $k(X)$, 
so it is an isomorphism from the function field of the image of $X$ to
$k(X)$, and therefore a birational isomorphism. 
\item{4.10} If $(x,y,w:z)\in A^2\times P^2$ is in $\phi^{-1}(Y)-$(exceptional
curve) then $y^2=x^3$, $xz=yw$, so $x^2(z^2-xw^2)=0$, 
so $z^2-xw^2=0$. Therefore the only possibility for this point
to lie on the exceptional curve $x=y=0$ is $(0,0,1:0)$. 
If $w=0$ then $x=0$ which is not possible, so we can define 
the map $f$ from $\bar Y$ to $A^1$ by $f(x,y,w:z)=z/w$. The inverse
takes $t$ to $(t^2,t^3, 1:t)$, so $\bar Y$ is isomorphic to $A^1$.

\bye
