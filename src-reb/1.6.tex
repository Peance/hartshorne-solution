\def\Z{{\bf Z}}
{\bf Hartshorne, Chapter 1.6} Answers to exercises. \hfill REB 1994
\item{6.1a} By 6.7, $Y$ is isomorphic to an open subset of some
projective space, and therefore to a proper open subset of $P^1$, and
therefore to some open subset of $A^1$.
\item{6.1b} We can assume $Y=A^1\backslash \{a_1,\ldots,a_n\}$.
Then $Y$ is isomorphic to the subset 
$y(x-a_1)\cdots(x-a_n)=0$ of $A^1$. 
\item{6.1c} Any element of $A(Y)$ can be written uniquely 
in the form $a(x-b_1)^{c_1}\cdots (x-b_n)^{c_n}$
with $c_i$ some integer, and $c_i$ positive if $b_i$
is not one of $a_1,\ldots, a_n$. Hence $A(Y)$ is
a U.F.D. with primes $x-b_i$ for $b_i\ne a_1,\ldots,a_n$. 
\item{6.2a} Singular points must satisfy 
$y^2-x^3+x=0$, $2y=0$, $-3x^2+1=0$, and if $k$ does not have characteristic
2 this implies $y=0$, $x=0,\pm 1$ which contradicts
$-3x^2+1=0$. The polynomial $y^2-x^3+x$ is irreducible,
because if it were reducible then its two factors
would intersect somewhere (possibly at infinity) 
and this point of intersection would be singular. (We should really 
also check that the curve has no singular points at infinity!)
As $Y$ is nonsingular, all points of $Y$ are normal, 
so $Y$ is normal, so $A(Y)$ is integrally closed. 
\item{6.2b} $k[x]$ is clearly a polynomial ring, 
and $y^2\in k[x]$ so $y$ is in the integral closure of 
$k[x]$. So $A$ is contained in  the integral closure of
$k[x]$, and is therefore equal to the integral closure
because $A$ is integrally closed. 
\item{6.2c} The automorphism $x\mapsto x$, $y\mapsto -y$
is an automorphism of $k[x,y]$ which maps the ideal
$(y^2-x^3+x)$ to itself and therefore induces an automorphism 
of $A$ (fixing $x$). Any element of $A$ can be written as
$yf(x)+g(x)$, so its norm is $(g(x)+yf(x))(g(x)-yf(x))
=g(x)^2-f(x)^2(x^3-x)\in k[x]$. The remaining properties
of $N(a)$ are trivial to check. 
\item{6.2d} If $a$ is a unit then $N(a)$ is also 
a unit (with inverse $N(1/a)$) so must be an element of $k$
as these are the only units in $k[x]$. 
But if $a=yf(x)+g(x)$, then its norm is
$g(x)^2-f(x)^2(x^3-x)$ and if $f$ is nonzero 
then the second term has odd degree while the first has
even degree so their sum cannot be a constant. Hence $f=0$,
and $g^2$ is a constant, so $a$ is a constant. 
To show that $A$ is not a UFD, note that $x$ and $y$
are irreducible (this follows easily by looking
at their norms $x^2$ and $x^3-x$ and noting that 
there are no elements whose norms is a degree 1 polynomial). 
But $x|y^2$ and $y$ is not a unit times $x$, so $A$ cannot be a UFD. 
\item{6.2e} $Y$ is clearly not $P^1$, and by exercise 6.1c
it is not $A^1$ minus a finite number of points, so
$Y$ is not rational. 
\item{6.3a} Map $A^2\backslash (0,0)$ to $P^1$ 
by $(x,y)\mapsto (x:y)$. 
\item{6.3b} Map $P^1\backslash \infty$ to $A^1$ in the
obvious way. 
\item{6.4} Any nonconstant rational map from 
$Y$ to $P^1$ induces $\phi^*$ from $k(x)$ to $k(Y)$, which is
injective.  Then every valuation ring of $k(x)$ can be extended to one
of $k(Y)$, so every point of $P^1$ is the image of a point of $Y$. For
every $p\in P^1$, $\phi^{-1}(P)$ is closed. If it was infinite it
would have to be all of $Y$ as the closure of any infinite subset of
$Y$ is $Y$, so the map $\phi$ would have to be constant.
\item{6.5} We know that $\bar X$ is a curve. If $x\in \bar X-X$ then
by 6.8 the map from $X$ to $X$ can be extended to 
a map from $x\cup X$ to $X$ which is impossible. (Alternatively 
this problem follows from the fact that the image of
any projective variety under a regular map is closed.)
\item{6.6a} The inverse of $x\mapsto (ax+b)/(cx+d)$
is $x\mapsto (dx-b)/(a-cx)$ if $ad-bc\ne 0$. 
\item{6.6b} Follows from corollary 6.12 (i) and (iii). 
\item{6.6c} Any automorphism of $k(x) $
maps $x$ to $f(x)/g(x)$ for some coprime polynomials $f$ and $g$,
and $x=h(f(x)/g(x))$ for some rational function $h$. Therefore
$f(x)/g(x)$ is not equal to $f(y)/g(y)$ if $x\ne y$. But if
$f$ or $g$ have degree greater than 1 then $g(y)a=f(y)$
will usually have more than one solution for $y$. Hence
$f$ and $g$ have degrees at most 1, and the result follows
from part (a).
\item{6.7} Any map from one curve to the other can be extended
to a map from $P^1$ to $P^1$, so the points $P_i$
must be mapped to the points $Q_j$, so $r=s$. The converse is true
if and only if $r\le 3$, because any set of at most 3 distinct points in $P^1$ 
can be mapped to any other set of the same size under $Aut(P^1)$, 
but this is not true for sets of 4 or more points. 
\bye

