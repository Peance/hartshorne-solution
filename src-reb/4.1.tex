%This is the solution to section 4.1 Hartshorne's algebraic geometry
\documentstyle[amssymb]{article}
\input amssym.def
\input amssym
\begin{document}

{\bf Solution to Math256a section IV.1}

1.1) Choose a positive integer $n$ larger than $deg(K)=2g-2$, and
$g$. By Riemann-Roch theorem, $l(nP)=n+1-g>1$. Thus there exists a
non-constant rational function $f$ over $X$ which has a pole at $P$ of
order $n>0$, and regular everywhere else. 

1.2) Induction on $r$. The case $r=1$ follows from the previous
exercise. Now assume there is a rational function $f$ having poles at
each of $P_1,\cdots, P_{r-1}$ of positive orders and regular
everywhere else. Since $f$ has no pole at $P_r$, we may let $n_r\geq
0$ be the coefficient of $P_r$ in $(f)$ (as a divisor). We may choose
a rational function $g$ which has a pole at $P_r$ of order $>n_r$ and
regular everywhere else (Cf. 1.1). Then $f\cdot g$ has poles precisely
at $P_1,
\cdots, P_r$ of positive orders.

1.5) Since $D$ is effective, $|K-D|\subseteq |K|$. Therefore
$l(K-D)\leq l(K)$. By Riemann-Roch theorem,
$l(D)=l(K-D)-g+deg(D)+1\leq l(K)-g+deg(D)+1=deg(D)+1$, since
$l(K)=g$. It follows that $dim(|D|)=l(D)-1\leq deg(D)$.
Proof showes that the equality holds iff $l(K-D)=l(K)=g$. 
If $D=0$, it is trivially true. If $g=0$, $deg(K)=-2$, so
$deg(K-D)<0$. Hence $l(K-D)=0$. It follows that $l(K-D)=l(K)=0$.

Conversely, suppose $l(K-D)=l(K)=g$. Suppose $D\not =0$. Let $P\in
Supp(D)$. Then $|K-D|\subseteq |K-P| \subseteq |K|$, thus $l(K-D)\leq
l(K-P) \leq l(K)$, and hence they are all equal. By Riemann-Roch,
$l(P)=l(K-P)+2-g=2$. Therefore there is a rational function $f$ with
one pole at $P$ of order $1$ and regualr everywhere else. This
function defines an isomorphism from $X$ to ${\Bbb P}^1$, thus
$g(X)=g({\Bbb P}^1)=0$.

1.6) Let $P$ be a point on $X$. By Riemann-Roch,
$$l((g+1)P)=l(K-(g+1)P)+(g+1)+1-g\geq 2.$$ Thus there exists a
rational function $f$ with a pole at $P$ of order $g+1$ and regular
everywhere else. This ration function induces a morphism $f:
X\rightarrow {\Bbb P}^1$ by sending $(g+1)P$ to ${\infty}\in {\Bbb
P}^1$. By II Prop. 6.9, $deg(f)=deg((g+1)P)=g+1$.

1.7) (a) It is clear that $deg(K)=2g-2=2$ and
$dim(|K|)=l(K)-1=1$. Suppose $P$ is a base point of $|K|$, then
$l(K-P)=l(K)=2$ by definition. By Riemann-Roch, $l(P)=2+2-2=2$. Thus
there exist a non-constant rational function $f$ with a pole at $P$ of
order $1$ and regular everywhere else. As we did before, $f$ defines
an isomorphism from $X$ to ${\Bbb P}^1$, contradiction since $X$ has
genus $2$ not $0$. Therefore $|K|$ has no base point.  Alternatively,
one may apply directly Prop 3.1 on page 307.  By II, 7.8.1, there is a
finite morphism $f: X\rightarrow {\Bbb P}^1$ with degree equal to
$deg(K)=2$. Therefore $X$ must be a hyperelliptic curve.

\end{document}



















































